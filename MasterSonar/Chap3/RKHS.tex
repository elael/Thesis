\section{RKHS - Reproducing Kernel Hilbert Space}

A \textit{Reproducing Kernel Hilbert Space}, RKHS for short, is a special kind
of Hibert Space of functions. In a RKHS, closeness in the sense of the metric is
actual pontwise proximity. That is to say, if two real-valued functions $f$ and
$g$ of a (non-empty) set $\mathcal{X}$ belong to a RKHS $\hs$ ($f,g \in \hs
\subset \mathbb{R}^\mathcal{X}$), then whenever $\hnorm{f-g}$ is small so is
$\norm{f(x)-g(x)}$ for all $x \in \mathcal{X}$\cite{berlinet2011reproducing}.

A more formal and useful characterization of a RKHS is consequece of studying
linear operators on Hibert Spaces. The evaluation functional \(\delta_x: \hs \to
\mathbb{R}\), \(\delta_x: f \to f(x)\) is easily seen as such: given \(f,g \in
\hs\) and \(a,b \in \mathbb{R}\), \(\delta_x(af+ag)= (af+ag)(x) = af(x)+ag(x) =
a\delta_x(f)+b\delta_x(g)\). When the evaluation functional is continuous on
$\hs$, $\hs$ is said to be a RKHS.

Although $L^2([a,b])$ is no a RKHS (it is not even a proper space of functions,
but instead a space of classes of equivalences of functions), its bandlimited
($L^2 \cap L^1$) version $\hs_{L^2} = {f\in L}$

Riesz representation theorem is an extension, for Hibert Spaces, of the
classical isomorphism between a finite vector space $\mathcal{V}$ and its dual
$\mathcal{V}^*$, the space of linear functions on $\mathcal{V}$. It states that
for every element $\varphi \in \hs^*$, where $\hs^*$ is the space
\textit{continuous} linear functionals from $\hs$ into $\mathbb{R}$ (dual
space), there exist a unique $f_\varphi \in \hs$, defined by:
\begin{equation*}
\varphi(g)= \hip{g}{f_\varphi}\qquad \forall g \in \hs
\end{equation*}

As consequece, the evaluation functional \(\delta_x\) has a representation on
\(\hs\) as \(k_x\), the reproducing property:
\begin{equation*}
f(x) = \delta_x(f) = \hip{f}{k_x}\qquad \forall g \in \hs
\end{equation*}

The important ideia of pointwise convergence can be recovered:

\begin{subequations}
\begin{align}
\norm{f(x)-g(x)} &= \norm{\delta_x(f-g)}\\
			  &= \norm{\hip{f-g}{k_x}}\\
			  &\leq \hnorm{f-g}\hnorm{k_x}\label{eq:cs_ineq}
\end{align}
\end{subequations}

Where Cauchy-Schwarz inequality was used on line \ref{eq:cs_ineq} and
\(\hnorm{k_x}\) acts as a scaling factor the closiness at each specific $x$.

The evaluation functional represented in $\hs$ as $k_x$ can be seen as a
function itself. As such, its evaluation at every point $y$ of $\mathcal{X}$
point contruct a two-variable function, the kernel:

\begin{equation}
k_x(y) = K(x,y) = \hip{k_x}{k_y}
\end{equation}

The kernel function \(K:\mathcal{X}\times\mathcal{X}\to\mathbb{R}\) is
symmetric (because this is the real case) and positive definite, as direct consequence of
inner product definition. The converse, however, is a result of the
Moore-Aronszajn theorem which says that for every symmetric positive definite
function $K(\parm,\parm)$ (kernel) on \(\mathcal{X}\times\mathcal{X}\) there is a unique
Hilbert space of functions on \(\mathcal{X}\) for which $K$ is a reproducing
kernel.

 

\citet{berlinet2011reproducing} -  Quase tudo sobre RKHS