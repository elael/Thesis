\section{RKHS - Reproducing Kernel Hilbert Space}

\subsection{The Evaluation Functional}
\label{ss:eval_func}
A \textit{Reproducing Kernel Hilbert Space}, RKHS for short, is a special kind
of Hibert Space of functions. In a RKHS, closeness in the sense of the metric is
actual pontwise proximity. That is to say, if two real-valued functions $f$ and
$g$ of a (non-empty) set $\mathcal{X}$ belong to a RKHS $\hs$ ($f,g \in \hs
\subset \mathbb{R}^\mathcal{X}$), then whenever $\hnorm{f-g}$ is small so is
$\abs{f(x)-g(x)}$ for all $x \in \mathcal{X}$\cite{berlinet2011reproducing}.

A more formal and useful characterization of a RKHS is consequence of studying
linear operators on Hibert Spaces. The evaluation functional \(\delta_x: \hs \to
\mathbb{R}\), \(\delta_x: f \to f(x)\) is easily seen as such: given \(f,g \in
\hs\) and \(a,b \in \mathbb{R}\), \(\delta_x(af+ag)= (af+ag)(x) = af(x)+ag(x) =
a\delta_x(f)+b\delta_x(g)\). When the evaluation functional is continuous on
$\hs$, $\hs$ is said to be a RKHS.

Although $L^2[a,b]$ is not a RKHS (it is not even a proper space of functions,
but instead a space of classes of equivalences of functions), its bandlimited
($L^2 \cap L^1$) version $PW_\pi := \left\{f\in L^2(\mathbb{R}) |~
\text{supp}~\mathcal{F}(f) \subseteq [-\pi,\pi]\right\}$, for example, has a
continuous evaluation functional. Here $\mathcal{F}:L^2(\mathbb{R})\to
L^2[-\pi,\pi]$ is the fourier transform\cite{trefethen1996finite}:

\begin{subequations}
\begin{equation}
\mathcal{F}(f) = \frac{1}{\sqrt{2\pi}}\int f(t)\mathrm{e}^{-i\omega t}\dif t
\qquad f \in L^2(\mathbb{R})
\end{equation}
\begin{equation}
\label{eq:inv_fourier}
\mathcal{F}^{-1}(\hat f) = \frac{1}{\sqrt{2\pi}}\int_{-\pi}^\pi
\hat f(\omega)\mathrm{e}^{i\omega t}\dif \omega \qquad \hat f \in L^2[-\pi,\pi]
\end{equation}
\end{subequations}

The proof of continuity for the evaluation functional on $PW_\pi$ relies on the
inverse Fourier transform \ref{eq:inv_fourier}, Cauchy-Schwarz's and Parseval's
theorems, used on equations \ref{eq:cont_four}, \ref{eq:cont_cs} and
\ref{eq:cont_par} respectively.
For $f,g \in PW_\pi$ it goes as follows:

\begin{subequations}
\begin{align}
\abs{\delta_x f-\delta_x g} &= \abs{f(x)-g(x)}\\
			  &=\abs{ \mathcal{F}^{-1}(\mathcal{F}(f-g)) }\\
			  &=\abs{ \ip{\mathcal{F}(f-g)}{\frac{\mathrm{e}^{-i\omega
			  t}}{\sqrt{2\pi}}}_{\scriptscriptstyle L^2[-\pi,\pi]} } \label{eq:cont_four}
			  \\
			  &\leq
			  \norm{f-g}_{\scriptscriptstyle
			  L^2[-\pi,\pi]}\norm{\frac{\mathrm{e}^{-i\omega x}}{\sqrt{2\pi}}}_{\scriptscriptstyle L^2[-\pi,\pi]} \label{eq:cont_cs} \\
			  &= \norm{f-g}_{\scriptscriptstyle L^2(\mathbb{R})} \label{eq:cont_par}
\end{align}
\end{subequations}

Other examples of RKHSs will be further explore as its relation to kernels are
developed.

\subsection{Reproducing Kernels}

Riesz representation theorem is an extension, for Hibert Spaces, of the
classical isomorphism between a finite vector space $\mathcal{V}$ and its dual
$\mathcal{V}^*$, the space of linear functions on $\mathcal{V}$. It states that
for every element $\phi \in \hs^*$, where $\hs^*$ is the space
\textit{continuous} linear functionals from $\hs$ into $\mathbb{R}$ (dual
space), there exist a unique $f_\phi \in \hs$, defined by:
\begin{equation*}
\phi(g)= \hip{g}{f_\phi}\qquad \forall g \in \hs
\end{equation*}

As consequence, the evaluation functional \(\delta_x\) has a representation on
\(\hs\) as \(k_x\), the reproducing property:
\begin{equation*}
f(x) = \delta_x(f) = \hip{f}{k_x}\qquad \forall g \in \hs
\end{equation*}

The important idea of pointwise convergence can be recovered:

\begin{subequations}
\begin{align}
\abs{f(x)-g(x)} &= \abs{\delta_x(f-g)}\\
			  &= \abs{\hip{f-g}{k_x}}\\
			  &\leq \hnorm{f-g}\hnorm{k_x}\label{eq:pw_cs_ineq}
\end{align}
\end{subequations}

Where Cauchy-Schwarz inequality was used on line \ref{eq:pw_cs_ineq} and
\(\hnorm{k_x}\) acts as a scaling factor the closeness at each specific $x$.

The evaluation functional represented in $\hs$ as $k_x$ can be seen as a
function itself. As such, its evaluation at every point $y$ of $\mathcal{X}$
point contruct a two-variable function, the kernel:

\begin{equation}
k_x(y) = K(x,y) = \hip{k_x}{k_y}
\end{equation}

The kernel function \(K:\mathcal{X}\times\mathcal{X}\to\mathbb{R}\) is
symmetric (because this is the real case) and positive definite as direct consequence of
inner product definition. The converse, however, is a result of the
Moore-Aronszajn theorem which says that for every symmetric positive definite
function $K(\parm,\parm)$ (kernel) on \(\mathcal{X}\times\mathcal{X}\) there is a unique
Hilbert space $\hs$ of functions (RKHS) on \(\mathcal{X}\) for which $K$ is a
reproducing kernel. The reproducing property of the kernel is:

\begin{equation}
f(x) = \hip{K(x,\parm)}{f} \qquad f \in \hs
\end{equation}

Examples of common used kernels:
\begin{itemize}
  \item Gaussian Kernel/Radial Basis Fucntion Kernel (RBF)
  \begin{equation}
  \label{eq:rbf}
  K(x,y) = \e^{-\gamma\norm{x-y}^2} \qquad \gamma \in \mathbb{R}^+
  \end{equation}
  \item Laplacian Kernel
  \begin{equation*}
  K(x,y) = \e^{-\lambda\norm{x-y}} \qquad \lambda \in \mathbb{R}^+
  \end{equation*}
  \item $PW_\pi$ Kernel (the bandlimited $L^2(\mathbb{R})$ space, see
  \ref{ss:eval_func})
  \begin{equation*}
  K(x,y) = \frac{\sin \pi(x-y)}{\pi(x-y)}
  \end{equation*}
  \item Linear Kernel
  \begin{equation*}
  K(x,y) = \ip{x}{y}
  \end{equation*}
  \item Polynomial Kernel
  \begin{equation*}
  K(x,y) = (\gamma\ip{x}{y}+1)^n \qquad \gamma \in \mathbb{R}, n \in
  \mathbb{N}^+
  \end{equation*}
\end{itemize}

Its possible to operate with kernels and generate new valid kernels. If
$K_1,K_2$ are kernels for $\hs_1$ and $\hs_2$, respectively, then for any
$\alpha,\beta \in \mathbb{R}_{\scriptstyle\geq0}$ it is possible to construct $K
= \alpha K_1 + \beta K_2$ as a kernel for the RKHS $\hs = \alpha\hs_1 +
\beta\hs_2 = \{ \alpha f_1 + \beta f_2 | f_1 \in \hs_1, f_2 \in
\hs_2\}$. Kernel products are valid even for functions acting on different sets,
that is, $K_1:\mathcal{X}\times\mathcal{X}\to\mathbb{R}$ and
$K_1:\mathcal{Y}\times\mathcal{Y}\to\mathbb{R}$ define
$K:(\mathcal{X}\times\mathcal{Y})\times(\mathcal{X}\times\mathcal{Y})\to\mathbb{R}$
as $K((x,y),(x',y'))=K_1(x,x')K_2(y,y')$ with the RKHS $\hs \cong \hs_1 \otimes
\hs_2$, having $\mathcal{X}=\mathcal{Y}$ as a special case.

\subsection{Feature Maps}

A feature map is a map $\varphi$ from a set $\mathcal{X}\neq\emptyset$ to a
Hilbert space $\hs$, the feature space. Any feature map can define a RKHS
through the kernel:
\begin{equation}
\label{eq:feature_kernel}
K(x,y)=\hip{\varphi(x)}{\varphi(y)} \qquad x,y \in \mathcal{X}
\end{equation}
For example, if $\mathcal{X}$ is already a RKHS by itself, then feature
map $\varphi(x)=x$ reconstruct the Linear Kernel. Also, a sequence of functions
$f_i \in \mathbb{R}^\mathcal{X}, \forall i \in \mathbb{N}$ that $\{f_i(x)\}
\in \ell^2, \forall x \in \mathcal{X}$ are themselves a feature map
$\varphi(x)=\{f_i(x)\}$ with kernel:
\begin{equation*}
K(x,y) = \sum_{i=1}^\infty f_i(x) f_i(y)
\end{equation*}

The converse, however, is not unique. Given a kernel there are multiple feature
maps that can generate it. A simple example is $K(x,y)=2xy$ with
$\mathcal{X}=\mathbb{R}$, this kernel can also be generated by:
\begin{equation*}
K(x,y) = \begin{bmatrix}x \\ x\end{bmatrix}\cdot\begin{bmatrix}y \\
y\end{bmatrix}
\end{equation*}

In the first case $\varphi_1(x)=x {\scriptstyle\sqrt{2}} \in \mathbb{R}$, while
in the second $\varphi_2(x)=\bigl[\begin{smallmatrix}x \\
x\end{smallmatrix}\bigr] \in \mathbb{R}^2$. Standard ways of finding and
approximating feature maps will be discussed in the next sections.