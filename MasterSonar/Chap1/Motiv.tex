\section{Purpose and Significance}
%\section{Motivation}

The mapping of underwater environments is not just a part of a SLAM system. It
might have importance on its own, it can be used for humans to visualize things
that could not be seeing otherwise. 

In the ROSA (\textit{Robô para Operações de Stoplogs Alagados}) project,
developed by LEAD/COPPETEC to ESBR, one of the goals is to make a reconstruction
of the hydroelectric power plant turbine entrance to spot any underwater debris
that could block the lowering of stoplogs, used to block the water flow, and
then cause delays and setbacks. When it happens for a stoplock to stuck the need
for a diver also incur in further human risk, so even then it is important to
know the surroundings.

On a developing perspective, besides the importance for the ROSA project, it has
characteristics that makes it appropriate for mapping. It has a lifting beam for
inserting the stoplogs into the water that can act as stable fixation point for
any sonar structure. So reducing the impact of poor a localization system on the
mapping. It also has a known and not-so-complex ground truth environment
(besides possible debris). Thus data collected there is suitable for testing the mapping
algorithm in a real world environment, as opposed to lab testing.
Although the noise environment might impair greatly the mapping, and subsequent
visual reconstruction.


It is also a integral part ROV's, where it gives feedback to the operator for
him to know where it is or/and what he is doing, especially because cameras do
not have a very useful range. And glaringly its automated counterpart (AUV's) as
a requisite for SLAM.