\section{Purpose and Significance}
%\section{Motivation}

%% SrJ: OK, drop the whole "SLAM" thing. Really, just mention it briefly in the
%% SrJ: beginning of the introduction and STOP
%%
%% SrJ: you're not doing SLAM, and mapping is important. Period.

%% SrJ: AFAIK, you won't run your algorithm on any ESBR data .... I wouldn't
% % SrJ: mention it here
% The mapping of underwater environments is not just a part of a SLAM system. It
% might have importance on its own, it can be used for humans to visualize things
% that could not be seeing otherwise. 

In the ROSA (\textit{Robô para Operações de Stoplogs Alagados}) project,
developed by LEAD/COPPETEC for ESBR (Energia Sustentável do Brasil), one of
the goals is to make a reconstruction of the hydroelectric power plant turbine
entrance. It should spot any underwater debris that could block the lowering of
stoplogs\footnote{long rectangular timber beams stacked to block water
flow.} and cause delays or even accidents.
Interestingly, the stoplog setup has characteristics that make it appropriate
for sonar mapping. It has a lifting beam for inserting stoplogs into
water that can act as stable fixation point for any sonar structure and
provide a good means of localization. Well placed high-end sonars could probably
scan such an environment, but they are expensive.

Mechanical imaging sonar is a more affordable type of sonar, however it suffers
from imprecise measurements caused by its wide sound beam. This works aims to
provide a method to map an environment using imaging sonars. It extends a recent
developed continuous map technique (Hilbert maps~\cite{ramos2016hilbert}) by
applying it to sonar responses. Continous maps are techniques that does not
discretize the space \textit{à priori} to create a map and, among them, Hilbert
maps is an easier to implement method that possess a high noise immunity.
It also implements a simulator with a trade-off between having simplifying assumptions and being as complete as possible for
imaging sonars, justifying the choices based on first physical principles and
other advanced simulation techniques.


% 
% 
% %% SrJ: this is really ... out of context
% It is also a integral part ROV's, where it gives feedback to the operator for
% him to know where it is or/and what he is doing, especially because cameras do
% %% SrJ: I don't get that last sentence.
% not have a very useful range. And glaringly its automated counterpart (AUV's) as
% a requisite for SLAM.
