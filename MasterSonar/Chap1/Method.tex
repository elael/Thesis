

\section{Methodology}
%\section{Metodology and Expected Results}
%But to get there it is important to describe some facts first.
 


% Using de aforementioned ideas, the main goal of the thesis reported here is to
% implement state-of-the-art techniques for mapping, using binary bayesian
% filters. Fuse all the data on a octree structure using a robotic framework,
% named ROCK, and implement a visualization for the reconstruct underwater map.
% The data source shall be a imaging sonar mounted on a pan-tilt unit, so to
% provide the sonar extra degrees of freedom.
% 
% Optimization of the visible surface for based on the expected response of a
% sonar beam in a given direction will also be attempted. It might supply
% information about the surface material, specifically its reflectance. A
% technique based on previously articles.


This work is divided into two parts.

The sonar model definition starts with a compilation on the description of
sonar, physical properties of sound waves in water, reflection, sonar
directional gain and sources of noise. Those are used to select a simulation
technique and model two different environment, a simple and a complex
structure, one to feed to the mapping algorithm and another to explore more
advanced acoustic features, e.g. multipath, directional gain. The simulation
results for both environments are then analyzed for those typical sonar
features.

The second part is related to mapping. An introductory chapter presents
the mathematical concepts used, followed by another with discussions on the
difficulties of 3D reconstruction and its methods. The latter includes description
of the most common and standard state-of-the-art techniques, with comments on
some alternative works, and deeper details of Hilbert maps. Hilbert maps
implementation and results, for one of the simulated environments, are
displayed at the end of the chapter.

% The second branch deal with the map filling over a discretized space, based on
% a Binary Bayesian Filter implementation. That is the standard state-of-the-art
% technology for mapping \cite{thrunprob}.


% The second branch deal with the map filling, basically the Binary Bayesian
% Filter implementation. A review on Bayesian Filtering is scheduled before the
% coding writing to be done on the robotics framework, ROCK. The implementation
% will make use of the Octotree data structure, through the Octomap library, to
% store the map.

% The integration of the sonar model with the Bayesian Filter give the means to
% process sonar data. So data acquired on the LNDC/UFRJ tank (Laboratório de
% Ensaios Não Destrutivos, Corrosão e Soldagem - which loosely translated means
% Non-Destructive Testing, Corrosion and Welding Laboratory) and on the Jirau
% Power Plant, by means of the ROSA/COPPETEC project, will be processed and
% compared to the tank and power plant entry layout, respectively.
%  
% In a more complex endeavor, which is the third branch, a theoretical derivation
% for the optimization of 2D surfaces embed on 3D environments will take place.
% It uses the sonar response as expectation and, instead of having a fixed
% relation between the measured environment and the sonar response, aim to better
% infer the underlying geometry of the surrounds and also retrieve some
% information about the material's reflectivity properties.
% This optimization algorithm will then be integrated into the Bayesian Filter. The
% same data processed by the combination ``Filter + Sonar Model'' shall now be
% processed by ``Filter + Optimization''.
%  
% Based on the literature, even with not much similar works, the reconstruction
% with \textit{a priori} sonar model will probably experience problems when
% reconstructing corners, shallow angle surfaces, very complex (intense multipath)
% or on highly noise environments. For the optimization, it shall encounter
% similar issues, but a less accentuated shallow angle quality degradation and an
% overall less blurry reconstruction.
