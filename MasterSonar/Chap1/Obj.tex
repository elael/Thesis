

\section{Objectives}

%% SrJ: try to be more bullet-pointy
%% SrJ: you already made your point about "that can be done in many ways" focus
%% SrJ: on what this thesis does.

The by-product of this work is a simulator and a mapping system. The simulator
will be able to receive a general environment and calculate the expected
response of a high-frequency imaging sonar, which technical details are based on
Tritech's Micron sonar. Its output shall exhibit common sonar features as
multipath effects, noise and beamwidth uncertainty. The response will be
presented as polar plot similar to those used by real sonars.

The mapping system will receive sonar measuremnts, typically the simulator's
output, and generate a Hibert map representation of the continuous occupancy map.
The map has to match to the simulator ground truth avoiding inconsistencies
caused by the sonar's wide beam. The Hilbert map representation is, technically,
just a vector, thus it will be displayed as its occupancy map, evaluated at 2D
plane cuts from the actual 3D environment.




% The general objective is to map an underwater environment.
% But that can be done in many ways, for this thesis the main goal is to implement
% optimization of the visible surface based on the expected response of a
% sonar beam in a given direction and fuse multiple views using binary bayesian
% filters. 
% 
% The output of the whole process will be a 3D human-redable map of the scanned
% environemnt. The optimization step might, also, supply information about the
% surface material, specifically its reflectance. A technique based on previously
% articles.
% 
%  Fuse all the data on a octree structure using a robotic framework,
% named ROCK, and implement a visualization for the reconstruct underwater map.
% The data source shall be a imaging sonar mounted on a pan-tilt unit, so to
% provide the sonar extra degrees of freedom.

%  But to get there it is important to describe some facts first.
%  
% As stated before, profiling sonar try to mimic laser scanner, but using sound.
% So most of the approaches on 3D underwater SLAM focus on applying laser scanner
% techniques to profiling sonars, e.g. point cloud reconstruction. On the other
% hand, imaging sonars are usually cheaper and gives more information per sound
% beam. So having the possibility of using imaging sonar and benefiting from its
% extra information is a win-win strategy.
% 
% Besides the definition of the sonar, one should carefully look into the meaning
% of mapping, because, it can be interpreted in different ways, depending on the
% context.
% Taking what they all have in common, it is possible to generically define
% mapping as being the process of received environmental data to characterize
% the surroundings. In the understanding of characterization is where the
% difference lays, how the environs are going to be represented is dependent on
% the application.
% 
% On a SLAM system there is no intrinsic need for a human readable map. In such a
% system, it may be more interesting to store the map information, the
% \"characterization\", only through its most representative features, if that is
% what matters for the localization procedure.
% 
% Still, if the map is to be seen by a human it should store more information
% about the environment, so that it can be displayed as a usual map, 3D or 2D
% depending on the case. This representation also guides how the data could be
% stored, e.g. if it wants to show a surface, it can be stored as an elevation map,
% or if one wants to see a 3D object it can be stored as a point cloud.
% 
% Using de aforementioned ideas, the main goal of the thesis reported here is to
% implement state-of-the-art techniques for mapping, using binary bayesian
% filters. Fuse all the data on a octree structure using a robotic framework,
% named ROCK, and implement a visualization for the reconstruct underwater map.
% The data source shall be a imaging sonar mounted on a pan-tilt unit, so to
% provide the sonar extra degrees of freedom.
% 
% Optimization of the visible surface for based on the expected response of a
% sonar beam in a given direction will also be attempted. It might supply
% information about the surface material, specifically its reflectance. A
% technique based on previously articles.
