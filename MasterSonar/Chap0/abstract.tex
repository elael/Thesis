% Motiva\c{c}\~ao
O mapeamento, \`as vezes como parte de um sistema SLAM, \'e um tema de pesquisa
ativo e tem solu\c{c}\~oes not\'aveis usando scanners a laser, mas a maioria
do mapeamento subaqu\'atico \'e focada em mapas 2D, que tratam o ambiente como
uma planta, ou mapas 2.5D do fundo do mar .

% Declara\c{c}\~ao do problema
A raz\~ao para a dificuldade do mapeamento subaqu\'atico origina-se no seu
sensor, i.e. sonares. Ele contrasta com lasers, usados fora d'\'agua, que
s\~ao sensores precisos de baixo ru\'ido. Al\'em do seu ru\'ido, os
\textit{imaging sonars} t\^em um feixe sonoro muito amplo e, com isso, efetuam
uma medi\c{c}\~ao volum\'etrica.

% Abordagem
Esta disserta\c{c}\~ao desenvolve um simulador para \textit{single-beam imaging
sonars} de alta freqü\^encia e \'e capaz de replicar os efeitos t\'ipicos de
\textit{multipath}, ganho direcional e ru\'ido de fundo em ambientes
arbitr\'arios.
O simulador implementa um m\'etodo baseado na teoria geom\'etrica de raios,
cujos detalhes de como esse \'e consequ\^encia direta da acustica
subaqu\'atica, sob suposi\c{c}\~ao de curto comprimento de onda, também são
expostos no texto.

A sa\'ida do simulador \'e passada para um algoritmo de recontrução de mapas
cont\'inuos baseado em \textit{Hilbert Maps}. \textit{Hilbert Maps} surgem como
uma técnica de aprendizado de máquina sobre espa\c{c}os de Hilbert, usando
\textit{feature maps}, aplicadas ao contexto de mapeamento. No entanto, a
incorpora\c{c}\~ao de uma resposta sonar em um tal mapa \'e uma
contribui\c{c}\~ao desse trabalho.

% Resultados Conclus\~ao
Uma compara\c{c}\~ao qualitativa entre a ambiente de referência passado ao
simulador e o mapa reconstru\'ido revela \textit{Hilbert maps} como uma t\'ecnica
promissora para mapeamento atráves de sensores ruidosos e, tamb\'em, aponta para
algumas caracter\'isticas do ambiente dif\'iceis de distinguir.