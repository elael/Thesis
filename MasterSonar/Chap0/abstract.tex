% Motiva\c{c}\~ao
O mapeamento, \`as vezes como parte de um sistema SLAM\abbrev{SLAM}{Simultaneous
Localization and Mapping}, \'e um tema de pesquisa
ativo e tem solu\c{c}\~oes not\'aveis usando scanners a laser, mas a maioria
do mapeamento subaqu\'atico \'e focada em mapas 2D\abbrev{2D}{Two dimensional},
que tratam o ambiente como uma planta, ou mapas 2.5D\abbrev{2.5D}{Pseudo-three
dimensional, Elevation map} do fundo do mar.

% Declara\c{c}\~ao do problema
A raz\~ao para a dificuldade do mapeamento subaqu\'atico origina-se no seu
sensor, i.e. sonares\abbrev{Sonar}{Sound Navigation And
Ranging}. Em contraste com lasers (LIDARs\abbrev{LIDAR}{Light
Detection And Ranging}), os sonares s\~ao sensores imprecisos e com alto n\'ivel de ru\'ido. Al\'em do seu ru\'ido, os
sonares do tipo \textit{imaging} t\^em um feixe sonoro muito amplo e, com isso,
efetuam uma medi\c{c}\~ao volum\'etrica, ou seja, sobre todo um volume.

% Abordagem
Na primeira parte dessa disserta\c{c}\~ao se desenvolve um simulador para
sonares do tipo \textit{imaging} de feixo único de alta freqü\^encia capaz de
replicar os efeitos t\'ipicos de multicaminho, ganho direcional e ru\'ido de fundo em ambientes
arbitr\'arios.
O simulador implementa um m\'etodo baseado na teoria geom\'etrica de raios, com
todo seu desenvolvimento partindo da ac\'ustica subaqu\'atica.

Na segunda parte dessa disserta\c{c}\~ao, o simulador \'e incorporado em um
algoritmo de reconstrução de mapas cont\'inuos baseado em \textit{Hilbert Maps}.
\textit{Hilbert Maps} surge como uma técnica de aprendizado de máquina sobre
espa\c{c}os de Hilbert, usando mapas de caracter\'isticas, aplicadas ao contexto
de mapeamento. A incorpora\c{c}\~ao de uma resposta de sonar em um tal
mapa \'e uma contribui\c{c}\~ao desse trabalho.

% Resultados Conclus\~ao
Uma compara\c{c}\~ao qualitativa entre o ambiente de referência fornecido ao
simulador e o mapa reconstru\'ido pela técnica proposta, revela \textit{Hilbert Maps}
como uma t\'ecnica promissora para mapeamento atráves de sensores ruidosos e, tamb\'em, aponta para algumas caracter\'isticas do ambiente dif\'iceis de se distinguir, e.g. cantos
e regiões não suaves.
