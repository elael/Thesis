Esta dissertação sobre simulação submarina 3D e mapeamento consiste em duas
partes uma para simulação e outra para mapeamento. Um sistema de mapeamento
geralmente vem junto com um algoritmo de localização no que é chamado de SLAM
(\textit{Simultaneous Localization and Mapping}). O SLAM é um tema de
investigação ativo e tem soluções notáveis utilizando scanners a laser, mas a
maior parte do SLAM submarino está focada em mapas 2D que tratam o ambiente como
corte bidimensional ou em mapas de elevação do fundo do mar.

A razão para a problemática do mapeamento subaquático, e portanto do SLAM, está
no seu sensor, isto é, sonares. Pois estes não tem a mesma qualidade de resposta
que laser que são sensores precisos de baixo ruído.

O sensor padrão, sonar, é composto de hidrofones que lhes permite interagir com
o som na água. Portanto, são capazes de emitir e receber som ondas,
semelhante a microfones e alto-falantes. Eles são divididos em sonares passivos e sonares
ativos. Sonares passivos podem ser aplicados para ouvir seus arredores
e interpretar os ruídos pelo seu espectro.

Enquanto isso, os sonares ativos emitem um feixe de som para perceber e medir o
ambiente pelo eco criado. Os sonares ativos são classificados em duas categorias
importantes, \textit{profilings} e \textit{imagings}, com base no ganho
direcional do seu feixe.

Os \emph{profilings} têm um feixe em forma de cone com uma abertura de cerca de
$ 1.7 $ graus, esse é o ponto de meia potência. São considerados como tendo uma
resposta semelhante a de um laser, apesar de trabalhar com ondas sonoras.
Entretanto, eles não correlacionam muito, uma vez que diferem no nível de ruído,
tempo de resposta e resolução espacial.

Os sonares de \textit{imagings} serão o foco deste trabalho, eles empregam um
feixe muito mais largo que os \emph{profilings}. Eles possuem cerca de $ 3 $
graus de abertura vertical, embora mantendo a mesma abertura na plano
horizontal.
Ele não tem localização precisa do alvo, mas um pouco mais informação geral.
Portanto, é capaz de inferir a presença de objetos abaixo ou acima do seu
plano horizontal.
Desse ponto de vista, cada feixe fornece informação sobre uma ampla area
ensonificada.
Assim, com fusão de múltiplos feixes de diferentes direções e ângulos, áreas de
sobreposição destes deverão fornecer um melhor esboço do ambiente do que
apenas usando respostas de sonar isoladas.

Outra classificação, além da mencionada acima, é a \textit{multi}- ou
\textit{single beam}. Por lançar vários feixes de uma vez, estes tem uma taxa de
resposta mais rápida e não precisa direcionar seu hydrofone antes de medir sua
próxima posição angular. Quando Trabalhando com sonar \textit{multibeam}, sua
desvantagem continua sendo seu valor de mercado. Contrastando com a opção mais
barata, os sonar \textit{single beam}, que será estudado nesta dissertação.