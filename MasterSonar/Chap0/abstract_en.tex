This dissertation on 3D underwater simulation and mapping consists of two pieces
of software that is the main objective of the research, one for simulation and
another for mapping. A mapping system usually comes together with a localization
algorithm in what is called a SLAM (\textit{Simultaneous Localization and
Mapping}). SLAM is an active topic of research and has remarkable solutions
using laser scanners,
% http://www.frc.ri.cmu.edu/~jizhang03/Publications/RSS_2014.pdf
% http://www.frc.ri.cmu.edu/~jizhang03/Publications/IROS_2014.pdf
but most of the underwater SLAM is focused on 2D maps treating the environment
as a floor plant or as 2.5D maps on the seafloor.

The reason for the problematic of underwater mapping, and thus SLAM, takes part
in its sensor, i.e. sonars. Contrasting to laser-based systems used outside
water which are precise low-noise sensors.

The standard sensor, sonar, is composed of hydrophones which allows them to
measure sound in water. Therefore, enabling them to emit and receive sound
waves, resembling microphones and speakers. They are divided in passive sonars
and active sonars. Passive sonars can be applied to listen its surroundings,
interpreting the sounds by their spectrum.

Meanwhile, active sonars emit a beam of sound in order to perceive  and measure
the environment by the echo created. Active sonars are classified in two
important categories, profiling and imaging, based on its beam directional gain.

Profilings have a narrow pencil shaped beam  with an aperture of about $1.7$
degrees, i.e. the half power point. It is meant to have a similar response of a
laser scanner, despite working with sound waves. At last, they do not correlate
much, since they differ greatly on noise, response time and
spatial resolution.

Imaging sonars will be the focus of this work, they  employ a much wider beam
than profiling sonar. It is used to having around $3$ degrees angle of aperture
in the vertical direction, although keeping its same angular resolution on the
horizontal plane.
It does not have provide precise localization of the target, but a rather more
general information. Therefore, being able to infer the presence of objects
below or above its horizontal plane.
From this point of view, each beam gives broad  information about the region it
ensonifies. Thus, fusing multiple beams from different directions and angles,
containing overlapping areas,  is expected to provide a better outline of the
environment than just using isolated sonar responses.

Another classification, besides the one mentioned above, is the multi- or single
beam. Firing multiple beams at once gives a faster rate, not requiring waiting
for a response before redirecting the beam to its next angular position. When
working with multibeam sonar, its downside remains its market value. Contrasting
to the cheaper option, the single beam sonar, which will be further studied in
this thesis.