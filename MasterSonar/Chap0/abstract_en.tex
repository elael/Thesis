This proposal refers to a system for 3D underwater mapping. It consists of a
hardware plus a software that is the main objective of the research. The
chosen hardware will constrain the software options and decisions on the
underling algorithm.

A mapping system usually comes together with a localization algorithm in what is
called a SLAM (\textit{Simultaneous Localization and Mapping}). SLAM is an
active topic of research and has remarkable solutions using laser scanners,
%http://www.frc.ri.cmu.edu/~jizhang03/Publications/RSS_2014.pdf
%http://www.frc.ri.cmu.edu/~jizhang03/Publications/IROS_2014.pdf
but most of the underwater SLAM is focused on 2D maps treating the environment
as a floor plant or as 2.5D maps of the seafloor.

The reason for the problematic of underwater mapping, and thus SLAM, in contrast
with laser based systems used outside water is mainly its sensor. While lasers
are precise low-noise sensors, sonars, which are the standard sensor for
underwater SLAM, are the opposite.

Sonars measures the sound on the water, its main parts are the hydrophones. They
can emit and receive sound waves just like microphones and headphones do in the
air. They can be specialized to listen the environment and interpret its sounds,
usually by their spectrum, which is the case for passive sonars. Or they can
emit a beam of sound and wait for the reception of the echo.

 When talking about active sonars, the ones that measures sound echo emitted
by itself, there are two important classes based on its beam directional gain:
Profiling and Imaging.

Profiling have a narrow pencil shaped beam, with an aperture of about $1.7$
degrees, i.e. the half power point. It is meant to have a similar response of
what is expected from a laser scanner, but working with sound waves. At the end
of the day, they does not correlate much because they differ greatly on noise,
response time and spatial resolution.

Imaging sonars will be the focus of this work, they make use of a much wider
than Profiling sonar fan shaped beam. It use to have around $3$ degrees angle of
aperture in the vertical direction, but keeping same angular resolution on the
horizontal plane.
It does not have to precisely aim to provide a localized information about the
target, but rather a more general information, being able to infer some information
about the presence of objects below or above its horizontal plane. From this
point of view, each beam gives a concentrated information about the region it
ensonifies, so fusing the information of multiple beams from different
directions and angles is expected to give a better outline of the environment
than just using isolated sonar responses.

Besides classification based on beam shape, there are others as multi- or single
beam. Firing multiple beams at once gives a faster rate, as one does not need to
wait for the response to arrive before redirecting the beam to the next angular
position. But the least expensive still single beam and is the one that is going
to be considered.