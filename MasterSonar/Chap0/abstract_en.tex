%Motivation 
Mapping, sometimes as part of a SLAM system, is an active topic of
research and has remarkable solutions using laser scanners, but most of the
underwater mapping is focused on 2D maps, treating the environment as a floor
plant, or on 2.5D maps of the seafloor.

% Problem statement
The reason for the problematic of underwater mapping originates in its sensor,
i.e. sonars. It contrasts to lasers (LIDARs), sonars are
unprecise high-noise sensors. Besides its noise, imaging sonars have a wide
sound beam effectuating a volumetric measurement.

%Approach
The first part of this dissertation develops a underwater simulator for
high-frequency single-beam imaging sonars capable of replicating multipath, directional
gain and typical noise effects on arbitrary environments. The simulation relies
on a ray theory based method and explanations of how this theory follows from
first principles under short-wavelegnth assumption are provided.

In the second part of this dissertation, the simulator is combined to a
continous map algorithm based on Hilbert Maps.
Hilbert maps arise as a machine learning technique over Hilbert spaces, using
features maps, applied to the mapping context. The embedding of a sonar response in
such a map is a contribution.

%Results
%Conclusion
A qualitative comparison between the simulator ground truth and the reconstucted
map reveal Hilbert maps as a promising technique to noisy sensor mapping and,
also, indicates some hard to distinguish characteristics of the surroundings,
e.g. corners and non smooth features.

% 
% 
% This dissertation on 3D underwater simulation and mapping has as result 
% 
%  consists of two pieces
% of software that is the main objective of the research, one for simulation and
% another for mapping. 
% 
% % % http://www.frc.ri.cmu.edu/~jizhang03/Publications/RSS_2014.pdf %
% % http://www.frc.ri.cmu.edu/~jizhang03/Publications/IROS_2014.pdf
% 
% % A mapping system usually comes together with a localization
% % algorithm in what is called a SLAM (\textit{Simultaneous Localization and
% % Mapping}). SLAM is an active topic of research and has remarkable solutions
% % using laser scanners,
% % % http://www.frc.ri.cmu.edu/~jizhang03/Publications/RSS_2014.pdf
% % % http://www.frc.ri.cmu.edu/~jizhang03/Publications/IROS_2014.pdf
% % but most of the underwater SLAM is focused on 2D maps treating the environment
% % as a floor plant or as 2.5D maps on the seafloor.
% 
% 
% 
% % The standard sensor, sonar, is composed of hydrophones which allows them to
% % measure sound in water. Therefore, enabling them to emit and receive sound
% % waves, resembling microphones and speakers. They are divided in passive sonars
% % and active sonars. Passive sonars can be applied to listen its surroundings,
% % interpreting the sounds by their spectrum.
% % 
% % Meanwhile, active sonars emit a beam of sound in order to perceive  and measure
% % the environment by the echo created. Active sonars are classified in two
% % important categories, profiling and imaging, based on its beam directional gain.
% % 
% % Profilings have a narrow pencil shaped beam  with an aperture of about $1.7$
% % degrees, i.e. the half power point. It is meant to have a similar response of a
% % laser scanner, despite working with sound waves. At last, they do not correlate
% % much, since they differ greatly on noise, response time and
% % spatial resolution.
% 
% % Imaging sonars will be the focus of this work, they  employ a much wider beam
% % than profiling sonar. It is used to having around $3$ degrees angle of aperture
% % in the vertical direction, although keeping its same angular resolution on the
% % horizontal plane.
% % It does not have provide precise localization of the target, but a rather more
% % general information. Therefore, being able to infer the presence of objects
% % below or above its horizontal plane.
% % From this point of view, each beam gives broad  information about the region it
% % ensonifies. Thus, fusing multiple beams from different directions and angles,
% % containing overlapping areas,  is expected to provide a better outline of the
% % environment than just using isolated sonar responses.
% % 
% % Another classification, besides the one mentioned above, is the multi- or single
% % beam. Firing multiple beams at once gives a faster rate, not requiring waiting
% % for a response before redirecting the beam to its next angular position. When
% % working with multibeam sonar, its downside remains its market value. Contrasting
% % to the cheaper option, the single beam sonar, which will be further studied in
% this thesis.