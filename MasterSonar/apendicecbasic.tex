\chapter{\label{cbasic}Conceitos B�sicos}

\section{Propriedades da fun��o tra�o}



\section{\label{apsec::fund}Subespa�os fundamentais}
\noindent{\bf Teorema fundamental de ortogonalidade: } O espa�o nulo $Nul({\bf A})$ e o espa�o-linha $Col({\bf A}^T)$ s�o ortogonais e subespa�os de ${\mathbb{R}^n}$. O espa�o nulo � esquerda $Nul({\bf A}^T)$ e o espa�o-coluna $Col({\bf A})$ s�o ortogonais e subespa�os de ${\mathbb{R}^m}$. %A Figura \ref{fig::subespacos}

\section{Decomposi��o de Valor Singular (SVD)}

A decomposi��o de valor singular, ou simplesmente SVD, constitui uma ferramenta extremamente poderosa para an�lise em �lgebra linear. Qualquer matriz ${\bf A}\!\in\!\mathbb{R}^{m \times n}$ pode ser fatorada em \cite{STRANG}
%
\begin{equation}
\label{eq::svd1}
{\bf A} = {\bf U}\,{\boldsymbol \Sigma}\,{\bf V}^T = \text{{\bf (ortogonal)(diagonal)(ortogonal)}}.
\end{equation}
%
Os $r$ valores singulares (denotados por $\sigma$) na diagonal de ${\boldsymbol \Sigma}\!\in\!\mathbb{R}^{m \times n}$ s�o as ra�zes quadradas de autovalores n�o nulos de ${\bf A}{\bf A}^T$ e ${\bf A}^T{\bf A}$. As colunas de ${\bf U}\!\in\!\mathbb{R}^{m \times m}$ s�o autovetores de ${\bf A}{\bf A}^T$ e as colunas de ${\bf V}\!\in\!\mathbb{R}^{n \times n}$ s�o autovetores de ${\bf A}^T{\bf A}$. As matrizes ${\bf U}$ e ${\bf V}$ fornecem bases ortonormais para todos os quatros subespa�os fundamentais apresentados em \ref{apsec::fund}:

\vskip 0.5cm
\begin{tabular}{lccl}
Primeiras & $r$ & colunas de ${\bf U}$: & {\bf espa�o-coluna} de ${\bf A}$ \\
�ltimas & $m-r$ & colunas de ${\bf U}$: & {\bf espa�o nulo � esquerda} de ${\bf A}$ \\
Primeiras & $r$ & colunas de ${\bf V}$: & {\bf espa�o-linha} de ${\bf A}$ \\
�ltimas & $n-r$ & colunas de ${\bf V}$: & {\bf espa�o-nulo} de ${\bf A}$ \\
\end{tabular}
\vskip 0.5cm

\noindent A partir de (\ref{eq::svd1}), tem-se
%
\begin{equation}
{\bf A}{\bf V} = {\bf U}\,{\boldsymbol \Sigma},
\end{equation}
%
e, portanto, quando a matriz ${\bf A}$ multiplica uma coluna ${\bf v}_j$ de ${\bf V}$, produz $\sigma_j$ vezes uma coluna de ${\bf U}$. %A raz�o $\sigma_{max}/\sigma_{min}$ � o n�mero de condicionamento de uma matriz $n$ por $n$ inversa e a disponibilidade dessa informa��o
