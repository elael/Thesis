\chapter{\label{chap::sing}Singularidades e m�todos alternativos}

A solu��o de (\ref{eq:u1}) pode ser computada  somente  quando a matriz Jacobiana possui posto completo. Entretanto, ela pode tornar-se sem sentido quando o manipulador est� em  uma  {\it configura��o singular} e neste caso o sistema  de  controle  (\ref{eq:sys}) possui equa��es linearmente dependentes. � importante ressaltar que a invers�o da matriz Jacobiana pode representar um grande incoveniente n�o apenas em uma configura��o singular mas  tamb�m  na  vizinhan�a  de  uma  singularidade, onde a matriz se torna mal condicionada, apresentando piv�s muito pequenos e resultando em grandes valores de velocidades das juntas. Neste cap�tulo � apresentada uma an�lise de configura��es singulares para manipuladores antropom�rficos, cinematicamente simples e bastante empregados por tal facilidade. Ser�o apresentados tamb�m dois algoritmos gerais para solucionar o problema de cinem�tica inversa na presen�a de singularidades. O algoritmo DLS, \cite{NH:86} e \cite{W:86}, utiliza um fator de amortecimento para tornar a invers�o melhor condicionada. Por sua vez, o algoritmo FIK, \cite{P:08}, apresenta uma proposta alternativa que n�o exige invers�o matricial, solucionando o problema sob o ponto de vista de controle, empregando um la�o de realimenta��o para minimizar a discrep�ncia entre as velocidades desejada e atual.