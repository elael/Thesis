\subsection{Solu��o para cinem�tica inversa}

\begin{frame}{Solu��o para cinem�tica inversa}
Como visto, a escolha
%
\begin{equation}
\dot{\bf q} = {\bf J}^{-1}\,{\boldsymbol \nu},
\end{equation}
% 
para ${\boldsymbol \Lambda} = {\boldsymbol \Lambda}^T > 0$ e ${\boldsymbol \nu}=\dot{\bf x}_d + {\boldsymbol \Lambda}{\bf e}$ garante a converg�ncia exponencial do erro ${\bf e}$ para ${\bf 0}$.

Para manipuladores redundantes ($m \leq n$), utiliza-se ${\bf J}^{\dagger} = {\bf J}^T({\bf J}{\bf J}^T)^{-1}$.

Proposta:

\begin{equation}
\label{eq::cinvini}
\dot{\bf q} = {\boldsymbol \Theta}\,{\boldsymbol \nu}
\end{equation}

onde ${\boldsymbol \Theta}\!\in\!{\mathbb R}^{n \times m}$ � a inversa filtrada da matriz Jacobiana ${\bf J}$ e

\begin{equation}
\dot{\boldsymbol \Theta} = -{\boldsymbol \Gamma}({\bf J}^T{\bf S}_r + {\bf S}_\ell{\bf J}^T)
\end{equation}
%
onde ${\boldsymbol \Gamma} = \gamma {\bf I} > 0$ (escalar).
\end{frame}