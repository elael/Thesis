\section{Sonar}

Throughout this thesis only one type of sonar is considered, the mechanical
imaging active sonar\ref{ss:avaible_models}. Sonars have a comum underlying
principle of operation, but vary greatly on aplication and hardware constituition.

Sonars are, in some sense, the acoustic analog of a camera. They use sound,
instead of light, to capture information about the environment. So, to better
undersand \textit{how} they operate and \textit{what} are they used for, it is
important to have a clear concept of sound.

\subsection{Physics of Sound}

The phenomenon that humans percieve as sound is a pressure wave that amplitude
excess the mean pressure of the medium \cite{FEYNMAN}. It can me referred to as
\textit{compressional} or \textit{longitudinal} waves, contrasting with
\textit{transversal waves}. The difference between these two kinds of waves
relies on the direction of the movement of the particles, being parallel or
perpendicular to the propagation of the wave, respectively\cite{BRUNEAU}.

On the particular, but usefull, conditions of low energy
phenomena\cite{Lefebvre} (with some other suitable requirements\footnote{A
perfect simple fluid in an initial state of stationary homogeneous equilibrium})
the pressure pertubation wave can be described as the \textit{D'Alembert
equation}:
 
\begin{equation}\label{eq:lambert}
\nabla^2 \Phi - \frac{1}{c^2_0}\frac{\partial^2}{\partial t^2} \Phi = 0
\end{equation}

Where $\Phi$ is the velocity potencial, a scalar field that helps describing the
sound propagation. Its relation to sound pressure is:

\[ p =  -\rho \frac{\partial}{\partial t}\Phi \]

Which can be directly described as:

\begin{equation} \label{eq:wave}
\nabla^2 p - \frac{1}{c^2_0}\frac{\partial^2}{\partial t^2} p = 0
\end{equation} 
 
Where $p$ is the pressure deviation from the mediums, $\rho$ the density, $c_0$
is the local sound speed and $\nabla^2$ stands for the Laplace operator. These equations are
only valid in free space (no source), but discrete variations of the medium are
treated as boundary conditions, giving origin to reflection and refraction.

Besides pressure, sound has another important derived property: intensity. Much
like the case of electromagnetic waves, sound intensity (or acoustic intensity)
measures the mean value of the sound energy flux (i.e. energy rate
per area):

\begin{equation}\label{eq:intensity_mean}
\vec{I} = \overline{p\vec{v}}
\end{equation}

Where $\vec{I}$ represents the \textit{acoustic intensity} vector, $\vec{v}$ the
\textit{acoustic velocity} (i.e. the velocity of a particle in the medium) and the
overline the mean over some time period. The \textit{acoustic velocity} can also
be derived from the velocity potencial $\Phi$ as:

\[ \vec{v} = \nabla \Phi\]

When considering a wave far from its source, solutions to the equation
\ref{eq:wave} give rise to a \textit{plane wave}( where the coherent wave front
propagate in a plane). It makes clear the relationship between $\vec{v}$ and
$p$:

\[ \vec{v} = \frac{p}{\rho c_0} \vec{n_0} \]

Where $\vec{n_0}$ is the unit normal vector to the wavefront. Pluging it back to
equation \ref{eq:intensity_mean}:

\begin{equation}\label{eq:intensity_pressure}
\vec{I} = \tfrac{1}{\rho c_0} \overline{p^2} \vec{n_0}
\end{equation}

This equation shows the proportionality between the \textit{acoustic
intensity} and the mean squared of the pressure. The inverse of the
proportionality constant $\rho c_0$ is called the \textit{characteristic
impendace} because it measures the degree of ``resistense to propagation'' of
the medium.

%By means of the same reasoning about the physical properties

 %(perpendicular to the direction of propagation)

Because the acoustic intensity (and related quantities) varies orders of
magnitude while propagating, it is commom to quantify it on a logarithmic scale,
specifically \textit{decibels} (dB)\cite{LURTON}:

\begin{equation}\label{eq:dB}
I_{dB} = 10~\log_{10}\left(\frac{I}{I_0}\right)
\end{equation} 

Here $I_{dB}$ is the intensity measured in \textit{decibels}, $I$ the intensity
value and $I_0$ a reference intensity values, usually defined somewhere near the
source. In the case of reflected/refracted wave, $I_0$ may also refere to
the intensity of the incoming wave. A direct relation between the magnitude of
intensity and pressure is found by applying equation \ref{eq:intensity_pressure}
on equation\ref{eq:dB}:

\begin{equation}\label{eq:dB}
I_{dB} = 20~\log_{10}\left(\frac{p_{\text{rms}}}{p_0}\right)
\end{equation}

Where $p_{\text{rms}}$ is the \textit{rms} (Root mean squared) value of the
wave's pressure ({\small $\sqrt{\text{\tiny \(\overline{p^2}\)}}$}) and $p_0$ is
a pressure value of reference.


\subsection{Sonar Principle of Operation}

The name Sonar (\textit{\underline{So}und \underline{N}avigation \underline{A}nd
\underline{R}anging}) was originally conceived for the tecnique that uses
acoustic waves on water for navigation, communication and detection, but
nowadays it is also used for the equipament that generate/receive these
sound waves.

The history is considered to have started on the year of 1490 through the
statemnt of Leonardo Da Vinci contained on the epigraph at the begining of this
chapter\cite{fahy1998fundamentals}. But that was birth of \textit{passive
sonar}'s technology, where the objetive is to listen (receive and process sound
waves) the noise from ships, animals and other objects in an attempt detect and
reconize its origin.

The concept of an \textit{active sonar}, one that emits a sound wave and detects
its return (as in figure \ref{fig:sonar_principle}), is much more recent. The
loss of the \textit{HMS Titanic} to a collision with an iceberg during its first
voyage on April 15 of 1912 \cite{histsonar} fostered the development of a sonar
to detect objects kilometers away. Also, during World War I,
Allied shipping losses to U-boat attacks further stimulated advances on tecniques for
unconvering of submerged enemies.

\begin{figure}
	\centering
		\begin{tikzpicture}[thick,scale=0.3, every node/.style={transform shape},line
	cap=round,line join=round,>=triangle 45,x=1.0cm,y=1.0cm] \clip(2.4174670085420336,-13.987437987397216) rectangle (50.355233369869445,8.357537118695749);
	\fill[color=ffqqqq,fill=ffqqqq,fill opacity=0.1] (5.,-4.) -- (5.,-1.) -- (8.,-1.) -- (10.,0.) -- (10.,-5.) -- (8.,-4.) -- cycle;
	\draw [color=qqqqff,fill=qqqqff,fill opacity=0.53] (46.,-2.5) circle (1.7879362227931317cm);
	\draw [color=ffqqqq] (5.,-4.)-- (5.,-1.);
	\draw [color=ffqqqq] (5.,-1.)-- (8.,-1.);
	\draw [color=ffqqqq] (8.,-1.)-- (10.,0.);
	\draw [color=ffqqqq] (10.,0.)-- (10.,-5.);
	\draw [color=ffqqqq] (10.,-5.)-- (8.,-4.);
	\draw [color=ffqqqq] (8.,-4.)-- (5.,-4.);
	\draw [shift={(5.,-2.5)},color=ffqqqq]  plot[domain=-0.46364760900080615:0.4636476090008061,variable=\t]({1.*8.375892603043697*cos(\t r)+0.*8.375892603043697*sin(\t r)},{0.*8.375892603043697*cos(\t r)+1.*8.375892603043697*sin(\t r)});
	\draw [color=ffffff] (4.,2.)-- (50.,2.);
	\draw [color=ffffff] (50.,2.)-- (50.,10.);
	\draw [color=ffffff] (50.,10.)-- (4.,10.);
	\draw [color=ffffff] (4.,10.)-- (4.,2.);
	\draw [color=ffffff] (44.,-8.)-- (4.,-8.);
	\draw [color=ffffff] (4.,-8.)-- (4.,-16.);
	\draw [color=ffffff] (4.,-16.)-- (42.20321483552447,-16.390821937761867);
	\draw [color=ffffff] (42.20321483552447,-16.390821937761867)-- (44.,-8.);
	\draw [shift={(46.,-2.5)},dash pattern=on 3pt off 3pt,color=qqqqff]  plot[domain=2.6779450445889874:3.6052402625905975,variable=\t]({1.*2.785722659294224*cos(\t r)+0.*2.785722659294224*sin(\t r)},{0.*2.785722659294224*cos(\t r)+1.*2.785722659294224*sin(\t r)});
	\draw [shift={(46.,-2.5)},dash pattern=on 3pt off 3pt,color=qqqqff]  plot[domain=2.6779450445889874:3.6052402625905993,variable=\t]({1.*5.571445318588448*cos(\t r)+0.*5.571445318588448*sin(\t r)},{0.*5.571445318588448*cos(\t r)+1.*5.571445318588448*sin(\t r)});
	\draw [shift={(46.,-2.5)},dash pattern=on 3pt off 3pt,color=qqqqff]  plot[domain=2.677945044588987:3.6052402625905997,variable=\t]({1.*8.357167977882664*cos(\t r)+0.*8.357167977882664*sin(\t r)},{0.*8.357167977882664*cos(\t r)+1.*8.357167977882664*sin(\t r)});
	\draw [shift={(46.,-2.5)},dash pattern=on 3pt off 3pt,color=qqqqff]  plot[domain=2.677945044588987:3.6052402625905993,variable=\t]({1.*11.142890637176889*cos(\t r)+0.*11.142890637176889*sin(\t r)},{0.*11.142890637176889*cos(\t r)+1.*11.142890637176889*sin(\t r)});
	\draw [shift={(46.,-2.5)},dash pattern=on 3pt off 3pt,color=qqqqff]  plot[domain=2.677945044588987:3.605240262590599,variable=\t]({1.*13.928613296471113*cos(\t r)+0.*13.928613296471113*sin(\t r)},{0.*13.928613296471113*cos(\t r)+1.*13.928613296471113*sin(\t r)});
	\draw [shift={(46.,-2.5)},dash pattern=on 3pt off 3pt,color=qqqqff]  plot[domain=2.677945044588987:3.6052402625905993,variable=\t]({1.*16.714335955765335*cos(\t r)+0.*16.714335955765335*sin(\t r)},{0.*16.714335955765335*cos(\t r)+1.*16.714335955765335*sin(\t r)});
	\draw [shift={(46.,-2.5)},dash pattern=on 3pt off 3pt,color=qqqqff]  plot[domain=2.677945044588987:3.6052402625905993,variable=\t]({1.*19.500058615059555*cos(\t r)+0.*19.500058615059555*sin(\t r)},{0.*19.500058615059555*cos(\t r)+1.*19.500058615059555*sin(\t r)});
	\draw [shift={(46.,-2.5)},dash pattern=on 3pt off 3pt,color=qqqqff]  plot[domain=2.677945044588987:3.6052402625905993,variable=\t]({1.*22.285781274353777*cos(\t r)+0.*22.285781274353777*sin(\t r)},{0.*22.285781274353777*cos(\t r)+1.*22.285781274353777*sin(\t r)});
	\draw [shift={(46.,-2.5)},dash pattern=on 3pt off 3pt,color=qqqqff]  plot[domain=2.677945044588987:3.605240262590599,variable=\t]({1.*25.071503933648003*cos(\t r)+0.*25.071503933648003*sin(\t r)},{0.*25.071503933648003*cos(\t r)+1.*25.071503933648003*sin(\t r)});
	\draw [shift={(46.,-2.5)},dash pattern=on 3pt off 3pt,color=qqqqff]  plot[domain=2.677945044588987:3.6052402625905993,variable=\t]({1.*27.857226592942222*cos(\t r)+0.*27.857226592942222*sin(\t r)},{0.*27.857226592942222*cos(\t r)+1.*27.857226592942222*sin(\t r)});
	\draw [shift={(5.,-2.5)},color=ffqqqq]  plot[domain=-0.46364760900080615:0.4636476090008061,variable=\t]({1.*11.16161526233792*cos(\t r)+0.*11.16161526233792*sin(\t r)},{0.*11.16161526233792*cos(\t r)+1.*11.16161526233792*sin(\t r)});
	\draw [shift={(5.,-2.5)},color=ffqqqq]  plot[domain=-0.46364760900080615:0.4636476090008061,variable=\t]({1.*13.947337921632139*cos(\t r)+0.*13.947337921632139*sin(\t r)},{0.*13.947337921632139*cos(\t r)+1.*13.947337921632139*sin(\t r)});
	\draw [shift={(5.,-2.5)},color=ffqqqq]  plot[domain=-0.46364760900080615:0.4636476090008061,variable=\t]({1.*16.733060580926363*cos(\t r)+0.*16.733060580926363*sin(\t r)},{0.*16.733060580926363*cos(\t r)+1.*16.733060580926363*sin(\t r)});
	\draw [shift={(5.,-2.5)},color=ffqqqq]  plot[domain=-0.46364760900080615:0.4636476090008061,variable=\t]({1.*19.51878324022059*cos(\t r)+0.*19.51878324022059*sin(\t r)},{0.*19.51878324022059*cos(\t r)+1.*19.51878324022059*sin(\t r)});
	\draw [shift={(5.,-2.5)},color=ffqqqq]  plot[domain=-0.46364760900080615:0.4636476090008061,variable=\t]({1.*22.304505899514808*cos(\t r)+0.*22.304505899514808*sin(\t r)},{0.*22.304505899514808*cos(\t r)+1.*22.304505899514808*sin(\t r)});
	\draw [shift={(5.,-2.5)},color=ffqqqq]  plot[domain=-0.46364760900080615:0.4636476090008061,variable=\t]({1.*25.090228558809027*cos(\t r)+0.*25.090228558809027*sin(\t r)},{0.*25.090228558809027*cos(\t r)+1.*25.090228558809027*sin(\t r)});
	\draw [shift={(5.,-2.5)},color=ffqqqq]  plot[domain=-0.46364760900080615:0.4636476090008061,variable=\t]({1.*27.875951218103253*cos(\t r)+0.*27.875951218103253*sin(\t r)},{0.*27.875951218103253*cos(\t r)+1.*27.875951218103253*sin(\t r)});
	\draw [shift={(5.,-2.5)},color=ffqqqq]  plot[domain=-0.46364760900080615:0.4636476090008061,variable=\t]({1.*30.661673877397476*cos(\t r)+0.*30.661673877397476*sin(\t r)},{0.*30.661673877397476*cos(\t r)+1.*30.661673877397476*sin(\t r)});
	\draw [shift={(5.,-2.5)},color=ffqqqq]  plot[domain=-0.46364760900080615:0.4636476090008061,variable=\t]({1.*33.4473965366917*cos(\t r)+0.*33.4473965366917*sin(\t r)},{0.*33.4473965366917*cos(\t r)+1.*33.4473965366917*sin(\t r)});
	\draw [shift={(5.,-2.5)},color=ffqqqq]  plot[domain=-0.46364760900080615:0.4636476090008061,variable=\t]({1.*36.23311919598592*cos(\t r)+0.*36.23311919598592*sin(\t r)},{0.*36.23311919598592*cos(\t r)+1.*36.23311919598592*sin(\t r)});
	\draw [shift={(5.,-2.5)},color=ffqqqq]  plot[domain=-0.46364760900080615:0.4636476090008061,variable=\t]({1.*39.018841855280144*cos(\t r)+0.*39.018841855280144*sin(\t r)},{0.*39.018841855280144*cos(\t r)+1.*39.018841855280144*sin(\t r)});
	\draw [color=ffqqqq] (24.,5.5)-- (27.,5.5);
	\draw [color=ffqqqq] (27.,5.5)-- (27.,5.);
	\draw [color=ffqqqq] (27.,5.)-- (27.5,5.5);
	\draw [color=ffqqqq] (27.5,5.5)-- (28.,6.);
	\draw [color=ffqqqq] (28.,6.)-- (27.5,6.5);
	\draw [color=ffqqqq] (27.5,6.5)-- (27.,7.);
	\draw [color=ffqqqq] (27.,7.)-- (27.,6.5);
	\draw [color=ffqqqq] (27.,6.5)-- (22.,6.5);
	\draw [color=ffqqqq] (22.,6.5)-- (22.,5.5);
	\draw [color=ffqqqq] (22.,5.5)-- (24.,5.5);
	\draw [color=xdxdff] (26.,-12.)-- (23.,-12.);
	\draw [color=xdxdff] (23.,-12.)-- (23.,-12.5);
	\draw [color=xdxdff] (23.,-12.5)-- (22.5,-12.);
	\draw [color=xdxdff] (22.5,-12.)-- (22.,-11.5);
	\draw [color=xdxdff] (22.,-11.5)-- (22.5,-11.);
	\draw [color=xdxdff] (22.5,-11.)-- (23.,-10.5);
	\draw [color=xdxdff] (23.,-10.5)-- (23.,-11.);
	\draw [color=xdxdff] (23.,-11.)-- (28.,-11.);
	\draw [color=xdxdff] (28.,-11.)-- (28.,-12.);
	\draw [color=xdxdff] (28.,-12.)-- (26.,-12.);
	\fill[color=ffffff,fill=ffffff,fill opacity=1.0] (4.,2.) -- (50.,2.) -- (50.,10.) -- (4.,10.) -- cycle;
	\fill[color=ffffff,fill=ffffff,fill opacity=1.0] (44.,-8.) -- (4.,-8.) -- (4.,-16.) -- (42.20321483552447,-16.390821937761867) -- cycle;
	\fill[color=ffqqqq,fill=ffqqqq,fill opacity=1.0] (24.,5.5) -- (27.,5.5) -- (27.,5.) -- (27.5,5.5) -- (28.,6.) -- (27.5,6.5) -- (27.,7.) -- (27.,6.5) -- (22.,6.5) -- (22.,5.5) -- cycle;
	\fill[color=xdxdff,fill=xdxdff,fill opacity=1.0] (26.,-12.) -- (23.,-12.) -- (23.,-12.5) -- (22.5,-12.) -- (22.,-11.5) -- (22.5,-11.) -- (23.,-10.5) -- (23.,-11.) -- (28.,-11.) -- (28.,-12.) -- cycle;
	\draw [color=ffqqqq](22.911186909624416,4.005463478846246) node[anchor=north west] {\textbf{Outgoing Wave}};
	\draw [color=qqqqff](23.17101220155573,-9.115713763685088) node[anchor=north west] {\textbf{Incoming Wave}};
	\end{tikzpicture}

	\caption{Depiction of the working principle of a \textit{active sonar}. The
	red speaker-like represented object represents the transducer, responsible for
	emiting and receiving the acoustic wave.}
	\label{fig:sonar_principle}
\end{figure}

Active sonars are ranging sensor and the way they infer distance is by measuring
the time between the emission and reception of a acoustic pulse (a time bounded
sound wave) like on \ref{fig:sonar_principle}. To be able to know space from the
delay, the mean sound velocity of the medium throughout the path traveled by the
pulse has to be known\cite{LURTON}:

\[ d = \frac{c_0 \Delta t}{2}  \]

Where $d$ is the distance between the source and the target, $c_0$ the mean
sound velocity, $\Delta t$ the delay between pulse emission and reception, and
the denominator $2$ is there because the time is measuring a two way trip of
the sound.

(underwater sound transducer)

% multipath
% chirp e resolução
% bearing


\subsection{Available Models}
\label{ss:avaible_models}
 
\cite{sonars:16} % tipos de sonar
