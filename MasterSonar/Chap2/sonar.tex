\section{Sonar}

Throughout this thesis only one type of sonar is considered, the mechanical
imaging active sonar. Sonars have a comum underlying principle of operation, but
vary greatly on aplication and hardware constituition.

Sonars are, in some sense, the acoustic analog of a camera. They use sound,
instead of light, to capture information about the environment. So, to better
undersand \textit{how} they operate and \textit{what} are they used for, it is
important to have a clear concept of sound.

\subsection{Physics of Sound}

The phenomenon that humans percieve as sound is a pressure wave that amplitude
excess the mean pressure of the medium \cite{FEYNMAN}. It can me referred to as
\textit{compressional} or \textit{longitudinal} waves, contrasting with
\textit{transversal waves}. The difference between these two kinds of waves
relies on the direction of the movement of the particles, being parallel or
perpendicular to the propagation of the wave, respectively\cite{BRUNEAU}.

On the particular, but usefull, conditions of low energy
phenomena\cite{Lefebvre} (with some other suitable requirements\footnote{A
perfect simple fluid in an initial state of stationary homogeneous equilibrium})
the pressure pertubation wave can be described as:
 
\begin{equation}\label{eq:laplace}
\nabla^2 p - \frac{1}{c^2_0}\frac{\partial^2}{\partial t^2} p = 0
\end{equation}
 
Where $p$ is the pressure deviation from the medium, $c_0$ is the local
sound speed and $\nabla^2$ stands for the Laplace operator. This equation is
only valid in free space (no source), but discrete variations of the medium are
treated as boundary conditions.

Besides pressure, sound has another important derived property: intensity. Much
like the case of electromagnetic waves, sound intensity (or acoustic intensity)
measures the mean value of the sound energy flux (i.e. energy rate
per area):

\begin{equation}\label{eq:intensity_mean}
\vec{I} = \overline{p\vec{v}}
\end{equation}

Where $\vec{I}$ represents the acoustic intensity vector and $\vec{v}$ the
acoustic velocity (i.e. the velocity of a particle in the medium).

 (perpendicular to the direction of propagation)


\subsection{Sonar Principle of Operation}

\cite{LURTON} % porra toda - principio de funcionamento

\subsection{Available Models}
 
\cite{sonars:16} % tipos de sonar
