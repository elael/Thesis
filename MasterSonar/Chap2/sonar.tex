\section{Sonar}

Throughout this thesis only one type of sonar is considered, the mechanical
imaging active sonar. Sonars have a comum underlying principle of operation, but
vary greatly on aplication and hardware constituition.

Sonars are, in some sense, the acoustic analog of a camera. They use sound,
instead of light, to capture information about the environment. So, to better
undersand \textit{how} they operate and \textit{what} are they used for, it is
important to have a clear concept of sound.

\subsection{Physics of Sound}

The phenomenon that humans percieve as sound is a pressure wave that excess the
mean pressure of the medium \cite{FEYNMAN}. It can me referred to as
\textit{compressional} or \textit{longitudinal} waves, contrasting with
\textit{transversal waves}. The difference between these two kinds of waves
relies on the direction of the movement of the particles, being parallel or
perpendicular to the propagation of the wave, respectively\cite{BRUNEAU}.





\subsection{Sonar Principle of Operation}

\cite{LURTON} % porra toda - principio de funcionamento

\subsection{Available Models}
 
\cite{sonars:16} % tipos de sonar
