
\section{Hilbert Space}

A Hilbert Space is a complete inner product space (\citet{HS-YN:11}). It is a
complete metric space with respect to the metric induced by its inner product
(which in turn can be thought indirectly by it's induced norm). A nice picture
is as a generalization of the Euclidean space. Which means intution works well,
in contrast with the broader concept of Banach Spaces, a complete normed space,
where the infinite dimensional case can be quite different from what one would
expect\footnote{Banach Space are complete metric spaces where the metric does
not come necesserally from a inner product. (See \citet{HS-HJNB:00})}.

An inner product space (\citet{HS-HJNB:00}) is a (possibly infinite dimensinal)
vector space $V$ over $\mathbb{C}$ (or $\mathbb{R}$ by restriction ), together
with a map (called the \textbf{inner product}):

\[  \langle\cdot,\cdot\rangle: V \times V \to \mathbb{C} \]

Satisfing the following properties, for all $x,y,z \in V$ and all $\mu, \lambda
\in \mathbb{C}$:

\begin{enumerate}[I]
  \item \(  \langle x,\lambda y + \mu z  \rangle = \lambda\langle	 x,y\rangle +
  \mu \langle x,z \rangle \) (linear in the second argument)
  \item \( \langle x,y \rangle = \overline{\langle y,x \rangle } \) (Hermitian
  symmetric)
  \item  \( \langle x,x \rangle \geq 0 \) and \( \langle x,x \rangle = 0
  \Leftrightarrow x = 0 \) (positive definite)
\end{enumerate}

A classical example of a inner product is the euclidean dot product.

Another important exemple is the inner product defined on the space C$[a,b]$ of
complex (or real) valued continuous functions on the interval $[a,b]$, defined,
for every $f$ and $g$ in C$[a,b]$ as:

\begin{equation}\label{eq::func_inner}
  \langle f,g\rangle = \int_a^b \overline{f(x)}g(x) \mathrm{d}x
\end{equation}

On any Hibert Space $H$ the norm induced by the inner product is:

\begin{equation}
  \| x \|_{_H} = \sqrt{ \langle x,x \rangle }
\end{equation}

where $x \in H$. And the subsequent metric is defined as:

\begin{equation}\label{eq::norm_metric}
  d_{_H}(x,y) = \| x - y \|_{_H}
\end{equation}

for any $x,y \in H$.

A vector space endowed with a inner product is a \textit{inner product space}
(aka. pre-Hilbert space). For it to be a Hibert Space it also has to be complete
with respect to the above metric. Completeness means that any Cauchy sequence
converges in this space (which provides a suitable framework to apply the tools
of calculus). A Cauchy sequence is a sequence where every term becomes
arbitrarily close to each other as the sequence progress (not only to term
right next to it). It can be formalised as the sequence
$x_1$,$x_2$,$x_3$,$\ldots$ on a metric space (with a metric $d( \cdot ,\cdot)$)
where:


\[ \forall \epsilon \in \mathbb{R}^+, ~\exists N \in \mathbb{Z}^+, ~\forall
n,m>N \Longrightarrow ~d(x_n,x_m)<\epsilon
\]

On a pre-Hilbert space, the metric is given by equation \ref{eq::norm_metric}.
If a metric space $M$ is complete, then every Cauchy sequence
($x_1$,$x_2$,$x_3$,$\ldots$) converges in that space:

\[ \exists x \in M, \forall \epsilon \in \mathbb{R}^+, ~\exists N \in \mathbb{Z}^+, ~\forall
n>N \Longrightarrow ~d(x_n,x)<\epsilon
\]
So,
\[ x = \lim_{n\to\infty} x_n \]

A complete metric space can be obtained from a pre-Hilbert space, by completion,
in the same way that $\mathbb{Q}$ is ``completed'' to make $\mathbb{R}$.
Although completeness is a technicality, it is easy to find exemples of
pre-Hibert spaces that lacks this property. The space of continuous functions
C$[a,b]$ with the inner product defined on \ref{eq::func_inner} gives an exemple
of pre-Hibert space that is not complete. For it to be a Hilbert space, the
space have to be extend to include some discontinous functions, as in the larger
set of Lebesgue mensurable\footnote{The Lebesgue mensurability of a, bounded
with compact support, function is a highly technical exigence and the existence
of a bounded non-Lebesgue mensurable set (which allow the construction of such a
function) is dependent on the axiomatic choice of the underlying set theory - it
can only be proven with the adition of the \textit{choice axiom} to the standard
ZF (Zermelo-Fraenkel) set of axioms. } functions that are square integrable
(with the Lebesgue integral).

Some exemples of Hilbert space are:

\begin{itemize}
  \item Any finite dimentional vector space over the field $\mathbb{R}$ or
  $\mathbb{C}$ with the standard dot product.
  \item The space $\ell^2$ of square-summable sequences of complex numbers, i.e.
  $(c_1,c_2,c_3,\ldots)$ with $c_i \in \mathbb{C}$ and $\sum_{i=1}^\infty |c_i|^2 <
  \infty$, is a Hilbert space with the inner product defined as: Given two sequeces
  $x=(x_1,x_2,x_3,\ldots)$ and $y=(y_1,y_2,y_3,\ldots)$, define $\langle x,y
  \rangle = \sum_{i=1}^\infty x_i \bar{y_i}$.
  \item Fourier series can be seen as the representation of a square-integrable
  function on the interval $[0,1]$ (member of $L^2([0,1])$) on the orthogonal
  basis $\{e^{2\pi i n \theta}\}_{n=1\ldots\infty}$ with the inner product given
  by \ref{eq::func_inner}.
\end{itemize}

