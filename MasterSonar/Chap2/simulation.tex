\section{Simulation}

Works on \textit{computacional ocean
acoustics}, the subfield of knowledge that explores the algorithms that model
the ocean as an acoustic medium, are well documented by
\citet{Etter2013}. Most of these works focus on very long range simulations,
with its most important features been the ocean floor and sub-bottom region.

This work aim to reconstruct and simulate near-range partially closed
environments, as those created by humans. The motivation for such a choice comes
from hydroeletrict power plant water intake, it is a corridor-like environment
with possible obstacles on the bottom. This kind of envirnoment is not well
covered on the underwater acoustics literature, as such, some of the simulations
techniques are borrowed from the closelly related are of \textit{room acoustics}.

When constructing a simulation one has to consider the tradeoff between
simplicity/perfomance/accuracy. There are several possible techniques with
different applications and assumptions, this chapter will cover the most
relevant ones and further explore \textit{ray theory}, which has been used for
the simulation presented here.

\subsection{Techniques Overview}

The idea behind sound simulation techniques is to solve the wave
equation (eq. \ref{eq:wave}) considering all the physical interfaces as boundary
conditions. The equation, however, cannot be analitically solved due to common
present discontinuities caused by occlusions, specular highlights and other
facts that result in large variations of the field over small regions of the
domain of integration\cite{funkhouser2003survey}.

The single most important reason that differentiate the several approches
described here is the \textit{wave frequency}. For high-frequency (where sound
speed do not very much in a wavelength scale) geometric methods (ray theory) are
justifiable and preferable (in the computational sense) \cite{urick1979}. In the
case of low/mid - frequency or in the presence of caustics\footnote{A region of
high constructive interference that geometrically gives a point of infinity
rays convegence.} other wave methods (e.g. finite elements, normal modes,
parabolic approximation) should be applied.

Instead of using the full wave equation, the methods work with a simplified
time-independ version. It starts by assuming that the equation
\ref{eq:wave}:

\[ \nabla^2 p - \frac{1}{c^2_0}\frac{\partial^2}{\partial t^2} p = 0 \]

Has a solution on the format:

\[ p(x,t) = a(x)\vartheta(t) \]

So that:
\[ \frac{\partial}{\partial t}a = 0 \]
\[ \nabla^2 \vartheta = 0 \]

Substituting it back and rearanging terms, gives:

\[ \frac{\nabla^2 a}{a} = \frac{1}{c_0^2\vartheta} \frac{\partial^2}{\partial
t^2}\vartheta  \]

As the expressions on both sides vary indepently, ie. l.h.s. varies with space
$s$ and r.h.s. with time $t$, they must be constant. As a matter of ajusting the
equation to fit its stantard parametrization, this constant is chosen to be
$-k^2$, then:

\[ \frac{\nabla^2 a}{a} = -k^2  \]
\[ \frac{1}{c_0^2\vartheta} \frac{d^2}{dt^2}\vartheta = -k^2  \]

Or

\begin{equation}
\label{eq:helmholtz}
(\nabla^2 -k^2)a = 0 
\end{equation}

\[ (\frac{d^2}{dt^2} + (kc_0)^2)\vartheta = 0  \]

With the apropriate definition of $\omega \equiv kc_0 $ the last equation,
becomes:

\begin{equation}
(\frac{d^2}{dt^2} + \omega^2)\vartheta = 0
\end{equation}

Equation \ref{eq:helmholtz} is known as the (homogeneous) \textit{Helmholtz
equation} and describe the time-independ part of the wave propagation. The
values of $k$ and $\omega$ can be phisically interpreted as the spatial and
tempral angular frequency of the wave. As the wave equation is a linear
equation superposition applies, it is resoanble to take into consideration
one wave frequency at a time and superpose all these harmonics by Fourier
synthesis\cite{Lefebvre}.



\subsubsection{FEM - Finite Element Method}\cite{funkhouser2003survey}
\subsubsection{Ray theory} \cite{torres2007modeling} \cite{danesh2013real}
\subsubsection{Normal modes} \cite{Etter2013}(?)
\subsubsection{Parabolic approximation} \cite{LURTON}

\subsection{Ray Theory}
\subsection{3D Enviroment Specifics}
\subsection{Results}
