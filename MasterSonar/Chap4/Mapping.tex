
\chapter{Mapping}

\renewcommand{\epigraphsize}{\footnotesize}
\epigraph{`Would you tell me, please, which way I ought to go from here?'

  `That depends a good deal on where you want to get to,' said the Cat.

  `I don't much care where--' said Alice.

  `Then it doesn't matter which way you go,' said the Cat. }{Alice's Adventures
  in Wonderland

  by Lewis Carroll}
  
Mapping is not a pre-defined concept, there are different ways to think about
mapping, all depends on one's need. The unifying idea is a method to fuse and
represent geometric information about a given environment. Nevertheless, the
meaning of how to represent which information is a consequence of the
application. Here, in this thesis, the objective is to have human-understandable
map yet with a probabilistic interpretation of a 3D environment. 

\section{Map Representation}
\citet{thrunprob} - Prob robotics
\subsection{Discrete Map}
\citet{Schwendner2013} - Discrete Map formulations
\citet{Pagac1998} - Evidencial Theory
\citet{Coiras2007} - 2.5D with properties
\subsection{Map of Features}
\citet{Ribas2006} - line extraction
\subsection{Continuous Map}
\citet{gan20093d} - Gaussian Process
\citet{ramos2016hilbert} - Hilbert Maps

\section{Inverse Sonar Model}
\citet{Moravec1985} - continuous inverse model (first hit)
\citet{thrunprob} - discrete model/Learn model

\section{Map Learning}
\citet{ramos2016hilbert} - Hilbert SGD 
\citet{bottou2012stochastic} - SGD tricks
\citet{tsuruoka2009stochastic} - Log Linear

\section{Implementation}
\subsection{Algorithm}
\subsection{Results}


% Directional Processing of
% Ultrasonic Arc Maps and
% its Comparison with
% Existing Techniques - Pe and Po com occupancy grid