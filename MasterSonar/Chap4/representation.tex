\section{Map Representation}
\label{s:map_rep}
Classically maps are binary functions over the space. The binary choices of
function's range stand for fullness or emptiness of a point.
Mapping (occupancy maps) is to construct a probability distribution over the set
of all maps~\cite{thrunprob}, i.e.
$\mathcal{M}_d=\{\coord{m}~|~\coord{m}:\mathbb{R}^d\to\{0,1\}\}$. The
cardinality of such a set $|\mathcal{M}_d|$ is too big\footnote{Side note: the
actual mathematical cardinality is $\beth_2=\mathbf{2}^\mathfrak{c}$ the
cardinality the power set of the continuum, where $\mathfrak{c}$ is the
cardinality of the continuum } to be tractable by exhaustion even considering
computational discretization (finite precision),i.e. by individually
assigning probabilities to each map.

Three approaches to this problem are presented here. One option is to consider
conditional independence of spatial points and discretize the space even
further, until it becomes tractable.
Another is to renounce the idea of binary maps and consider maps to be a collection of predefined
objects. And lastly, keep the concept of binary maps, but to consider some
restriction on the space of functions from which the map $\coord{m}$ is drawn.

Despite their differences, all approaches describe marginal probabilities
instead of the probability for an actual map $\coord{m}$. For binary maps,
that means $\Pr(\coord{m}(x_i)=1)$ and not $\Pr(\coord{m}=m_i)$ and, when
considering maps as collection of objects, it is
$\Pr(\mathfrak{s}(\mathbf{l}_n)=\mathfrak{s}_i)$, where $\mathfrak{s}(\parm)$ is
the defining properties of a object $\mathbf{l}_n$, for example point and angle
in the case of a line.

\subsection{Discrete Map}

Discrete maps are also denominated \textit{grid maps} because when discretizing
each axis of a $\mathbb{R}^d$ space the result is necessarily a grid. Originally
developed for 2D maps~\cite{thrunprob}, they were later extended to 3D maps
in different ways.

\subsubsection{3D grids}
\label{sss:3dgrid}
The first obvious extension was a 3D grid of cubes by discretizing a range
of each direction on $N$ elements. Reasoning that each cube still full
or empty, the set of possible maps on a $N_1\times N_2\times
N_3$ grid $\mathbf{D}$ is
$\bar{\mathcal{M}}_d=\{\coord{m}~|~\coord{m}:\mathbf{D}\to\{0,1\}\}$. The
cardinality of $|\bar{\mathcal{M}}_d|=2^{N_1\cdot N_2\cdot
N_3}$ is too big to store the probability of every element. 

The simplifying assumption for 3D grid is the conditional
independence of grid
elements $\coord{m}(d_i)$ and $\coord{m}(d_j)$ for $d_i\neq d_j \in \mathbf{D}$
on the sensors measurements $\mathbf{z}_n$.
Therefore, the probability of a map become the product of the marginals:
\begin{equation*}
\Pr(\coord{m}=m_i~|~\mathbf{z}_n)=\prod_{d\in\mathbf{D}}\Pr(\coord{m}(d)=m_i(d)~|~\mathbf{z}_n)
\end{equation*}

Writing marginals as $p_n(d) = \Pr(\coord{m}(d)=1~|~\mathbf{z}_n)$ keeps same
information because $\Pr(\coord{m}(d)=m_i(d)~|~\mathbf{z}_n)$ equals $p_n(d)$ if
$m_i(d)=1$ and $1-p_n(d)$ otherwise. The advantage is that it makes clearer that
they can be stored and updated independently, and also that the number of stored
elements is $|\mathbf{D}|=N_1\cdot N_2\cdot N_3$\footnote{e.g. if
$N_1=N_2=N_3=200$ for a 5cm resolution on cube with 10m edge,
$|\mathbf{D}|=8,000,000$ already.}.
That might still be a lot, but with clever memory implementations like
Octomaps~\cite{hornung2013octomap}, it can be manageable.

Marginal probability computation on each cube is a direct application of Bayes
rule\footnote{$\Pr(A~|~B)=\frac{\Pr(B~|~A)\Pr(A)}{\Pr(B)}$} (a Bayes filter)
with a log-odds representation for better faster
computation, known as occupancy in this context~\cite{thrunprob}:

\begin{equation}
\label{eq:gridoccupancy}
l_n(d) = l_{n-1}(d)+ \text{\textbf{inverse\_sensor}}(d,z_n) - l_0(d)
\end{equation}

Here $l_n(d)$ is the log-odds representation of $p_n(d)$, the nth estimate after
all previous $n$ sensor measurements $\mathbf{z}_n$, including the last one
$z_n$.

\begin{equation*}
l_n(d) = \log\frac{p_n(d)}{1-p_n(d)} 
\end{equation*}

The \textit{prior} of the occupancy is $l_0(d)$, log-odds of the \textit{prior}
probability $p_0(d)$. Defining
$\text{\textbf{inverse\_sensor}}(d,z_n)=\Pr(\coord{m}(d)=1~|~z_n)$, it is the
probability of fullness for a grid element $d$ given \textbf{only} the last
measuremnt $z_n$, it can be interpreted as inference from the sensor response,
justifying the name.

\citet{thrunprob} suggests to abandon independecy between grid elements. To
archive that, he employs a forward sensor model, instead of an
$\text{\textbf{inverse\_sensor}}$, and uses optimization algorithm Expectation
Maximization (EM) on the marginal to find the best fit. It is successful on
solving ``conflicts'' between sonar responses when, because of a wide beamwidth,
the same region appears to be full and empty depending on viewpoint.

Another, not so well explored, approach for calculating the marginal probability
for grid elements comes from Evidential Theory. Evidential Theory, a.k.a.
Dempster–Shafer theory (DST), is a mathematical theory of evidence, assigning
``probabilities'' (belief mass) to all non-empty elements of the power set of
events. On the binary $\{0,1\}$ case, the three non-empty subsets are
${0}$,${1}$,${0,1}$ standing for evidence of emptiness, fullness or both, which
``probabilities'' add to one.
Consequently yielding to two maps, one for fullness other for emptiness. In DTS
the actual probability (in the classical sense) appears as lower and upper
bounds (Plausability and Belief), allowing ignorance to be modeled adequately.
The 2D case was explored by \citet{Pagac1998}, their article also further
describes DTS.

Grid based algorithms on 3D environments suffer from their high number of grid
elements, the next model try to avoid this problem.

\subsubsection{Elevation Maps}

In an attempt to keep the grid to a reasonable size, elevation maps, or 2.5D
maps, keep the discretization only on the 2 horizontal dimensions. The third
dimension is represented as a height value assigned to each 2D discretization.

Some work on seabed reconstruction using sonars has been done by
\citet{Coiras2007,Coiras2009}. They attempt to map by reconstructing a
2.5D surface through optimization on the height value of each grid element. That
leads to information gain on local surface's reflectivity, an indication of its
composition. However, the expectation-maximization method does not leave a
direct probabilistic interpretation for the values.

Although elevation maps reduce memory requirements by not discretizing on
the vertical axis, its elevation value is unique for each grid element. As such,
it is not able to represent objects above the floor level,e.g. ceil, trees,
caves, etc. That is adressed in the next option.

\subsubsection{Multi Layer Surface - MLS}

As a compromise between the last two solution, grid and elevation maps, Multi
Layer Surface (MLS) was developed. It originates as elevation maps, but instead
of having only one height per grid element, it splits into many layers of
varying height, called \textit{surface patches}.

When MLS was first proposed by \citet{triebel2006multi}, each surface patch of
each grid element stored statistics as mean height and standard deviation. It
was soon realized that a flat horizontal plane for a grid and a preferential
vertical direction could be an issue for well describing statistical knowledge
of the environment. Some extensions have already been suggested to address this
matter, e.g. works of \citet{rivadeneyra2011probabilistic} and
\citet{Schwendner2013}.

\subsection{Map of Features}
\label{ss:mapsoffeatures}
A map can, in certain situations, be approximated by a collection of geometric
objects, viz. line segments, circles, etc. This is specially true for structured
environments~\cite{Ribas2006} where walls and flat surfaces are easily found.

The 2D simplification, only considering the horizontal plan, was explored for
underwater SLAM with Imagins Sonars by \citet{ribas2010underwater}. They used
lines to represent an environment, with endpoints only for display purpose. Line
extraction relied on polar the parametric space of lines (angle and distance to
the origin) as on Hough Voting and extensions.

Besides only applied to 2D, a particular downside of the approach is its little
generality. Generic, unstructured or complex environments are unhandleable.


\subsection{Continuous Map}

There is no need for discretization on space if some restrictions are applied to
the space of function of maps. Formally that means
$\tilde{\mathcal{M}}_d\subseteq\mathcal{V}\cap\mathcal{M}_d$, where $\mathcal{V}$
is some restricted space of functions, e.g. continuous compact supported, and
$\mathcal{M}_d$ as defined on section \ref{s:map_rep}. In practice the
restriction is not directly applied to the space of functions, instead it is
enforced on the probability distribution, such that functions outside
$\tilde{\mathcal{M}}_d$ have zero probability.

A marginal probability for continuous map evaluates at every point in
$\mathbb{R}^d$, not in some discretized space as for 3D grids presented in
section \ref{sss:3dgrid}. The marginals shall then be written as
$p(x)=\Pr(\coord{m}(x)=1)$ for $x\in\mathbb{R}^d$.

\subsubsection{Gaussian
Process Occupancy Maps - GPOM}

Possibly the first succesful attempt to have a continuous map was the Gaussian
Process Occupancy Maps (GPOM) by \citet{o2012gaussian}, as noted by the author
there were previous attemps, but they lacked computability or did not truly
represent occupancy. GPOM's method apply learning techniques, Gaussian
Process, to estimate the best marginal for a family of functions
$p(x)=\Phi(\parm)$, where $\Phi$ is the cumulative unit Gaussian. Details of the
learning process are numerous and complex, for the interested reader it is
suggested to check the original paper. GPOM was already applied to 3D
environments for path planning of a 6DOF Rotary Unmanned Aerial Vehicle (RUAV)
using laser
sensors\footnote{$\Phi(x)=\frac{1}{\sqrt{2\pi}}\displaystyle\int_{-\infty}^x \e^{-\nicefrac{t^2}{2}}dt$}~\cite{gan20093d}.

Following ideas from GPOM, Hilbert Maps were proposed as a less computationally
expensive alternative for continuous maps. It is also able to naturally account
for spatial correlations and handle noise, but while GPOM scales cubically with
data size, Hilbert Maps can be updated in linear time. Hilbert Maps is yet much
simpler to implement and it was the chosen alternative for sonar reconstruction
on this work. It suits well the task, as most of imaging sonar's drawbacks are
its noise and wide beamwidth.

\subsubsection{Hilbert Maps}
\label{sss:hilbertcontinuousmap}
Hilbert Maps is a recent development by \citet{ramos2016hilbert}, from 2016. It
was implemented on the 2D case, on a laser sensor setting. This thesis applies
the method to 3D environment with sonar, imaging sonar especially. Most of the
mathematical machinery necessary was described in chapter \ref{c:math}, however
some details still to be presented.

Marginal probabilities are assumed to be linear logistic as in equation
\ref{eq:hslogistic}:
\begin{equation*}
p(x;\coord{w}) = \frac{1}{1+\exp(-\hip{\coord{w}}{\varphi(x)})}
\end{equation*}

Where $\coord{w}, \varphi(x)\in\hs$, $\hs$ a Hilbert space that generates a RKHS
with kernel $K(x,y)=\hip{\varphi(x)}{\varphi(y)}$. As long range correlations
are not expected, a suitable kernel choice is the RBF defined on
equation \ref{eq:rbf}:
\begin{equation*}
K(x,y) = \e^{-\gamma\norm{x-y}^2} \qquad \gamma \in \mathbb{R}^+
\end{equation*}

The features $\varphi(x)$ are approximated as Nystr\"om features, other
methods did not show results as good~\cite{ramos2016hilbert}. Approximations
were discussed in section \ref{ss:reg_hs}. Nystr\"om features are sample
dependent, given set of $n$ ``inducing" points $\nu_i\in\mathbb{R}^d, \quad
i=1\ldots n$ chosen (possibly random) on the region being mapped, it is
calculated as:
\begin{equation}
\label{eq:nystrom}
\hat\varphi(x) = \sqrt{\mathbf{\Lambda}}\mathbf{Q}^T\hat K(x)
\end{equation}
 
 Where $\mathbf{G}=\mathbf{Q}\mathbf{\Lambda}\mathbf{Q}^T$ is the
 spectral decomposition of the Gram matrix of the ``inducing" points,
 $\mathbf{G}_{ij}=K(\nu_i,\nu_j)$, and ${\displaystyle\hat K}_i(x) = K(x,\nu_i)$
 is a column vector. After this finite approximations,
 $\coord{w}\in\mathbb{R}^n$ and equation \ref{eq:hslogistic} becomes:
\begin{equation}
\label{eq:hsmarg}
p(x;\coord{w}) = \frac{1}{1+\exp(-\coord{w}\cdot \hat\varphi(x))}
\end{equation}
 
 And the negative log likehood - NLL - regression, from equation \ref{eq:hsnll}
 becomes:
\begin{equation}
\label{eq:finitenll}
\text{NLL}_{\text{reg}}(\coord{w}) =  \sum_{i=1}^n \log(1+\exp(-y_i~
\coord{w}\cdot\hat\varphi(x_i))) +\lambda\mathcal{S}(\coord{w})
\end{equation}
 
 Where $(y_i,x_i)$ are the sample points and $\mathcal{S}(\parm)$ an
 elastic-net regularizer, described on the context of logistic regression of
 section \ref{ss:blr}:
  
\begin{equation}
\label{eq:elasticnet}
\mathcal{S}(\coord{w})=\alpha_1\norm{\coord{w}}_1 + \alpha_2\norm{\coord{w}}_2^2
\end{equation}

With $\alpha_1+\alpha_2=1$, $\alpha_1,\alpha_2\in[0,1]$.

Minimization of $\text{NLL}_{\text{reg}}(\coord{w})$ gives a representation of
the map as a $n$ parameter vector $\coord{w}$. It can then be evaluated at any
point with equation \ref{eq:hsmarg}. The minimization step is solved
iteractively as a learning processed, discussed further ahead, which is naturally an online
process.
 
 
