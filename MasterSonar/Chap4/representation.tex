\section{Map Representation}

Classically maps are binary functions over the space. The
binary choices of function's range stand for fullness or emptyness of a point.
Mapping is to construct a probability distribution over the set of all maps~\cite{thrunprob},
i.e.
$\mathcal{M}_d=\{\coord{m}~|~\coord{m}:\mathbb{R}^d\to\{0,1\}\}$. The
cardinality of such a set
$|\mathcal{M}_d|=\beth_2$ is too big\footnote{$\beth_2=\mathbf{2}^\mathfrak{c}$
the cardinality the power set of the continuum, where $\mathfrak{c}$ is the
cardinality of the continuum } to be tractable even considering computational discretization (finite precision).

Three approaches to this problem are presented here. One option is to discretize
the space even further than the computational limit. Another is to renounce
the idea of binary maps and consider maps to be a collection of predefined
objects. And lastly, keep the concept of binary maps, yet to consider some
restriction on the space of functions from which the map $\coord{m}$ is drawn.

Despite their differences, all approaches describe marginal probabilities
instead of the probability for an actual map $\coord{m}$. For binary maps,
that means $\Pr(\coord{m}(x_i)=1)$ and not $\Pr(\coord{m}=m_i)$ and, when
considering maps as collection of objects, it is
$\Pr(\mathfrak{s}(\mathbf{l}_n)=\mathfrak{s}_i)$, where $\mathfrak{s}(\parm)$ is
the defining properties of a object $\mathbf{l}_n$, for example point and angle
in the case of a line.

\subsection{Discrete Map}

Discrete maps are also denominated \textit{grid maps} because when discretizing
each axis of a $\mathbb{R}^d$ space the result is necessarily a grid. Originally
developed for 2D maps~\cite{thrunprob}, their where later extended to 3D maps
in different ways.

\subsubsection{3D grids}

The first obvious extension was a 3D grid of cubes by discretizing a range
of each direction on $N$ elements. Reasoning that each cube still full
or empty, the set of possible maps on a $N_1\times N_2\times
N_3$ grid $\mathbf{D}$ is
$\bar{\mathcal{M}}_d=\{\coord{m}~|~\coord{m}:\mathbf{D}\to\{0,1\}\}$. The
cardinality of $|\bar{\mathcal{M}}_d|=2^{N_1\times N_2\times
N_3}$ is too big to store the probability of every element. 

The simplifying assumption for 3D grid is the conditional
independece of grid
elements $\coord{m}(d_i)$ and $\coord{m}(d_j)$ for $d_i\neq d_j \in \mathbf{D}$
on the sensors measurements $\mathbf{z}$.
Therefore, the probability of a map become the product of the marginals:
\begin{equation*}
\Pr(\coord{m}=m_i~|~\mathbf{z})=\prod_{d\in\mathbf{D}}\Pr(\coord{m}(d)=m_i(d)~|~\mathbf{z})
\end{equation*}

Writing marginals as $p(d) = \Pr(\coord{m}(d)=1~|~\mathbf{z})$ keeps same
information because $\Pr(\coord{m}(d)=m_i(d)~|~\mathbf{z})$ equals $p(d)$ if
$m_i(d)=1$ and $1-p(d)$ otherwise. The advantage is that it makes clearer that
they can be stored and updated independently, and also that the number of stored
elements is $|\mathbf{D}|=N_1\times N_2\times N_3$\footnote{e.g. if
$N_1=N_2=N_3=200$ for a 5cm resolution on cube with 10m edge,
$|\mathbf{D}|=8,000,000$ already.}.
That might still be a lot, but with clever memory implementations like
Octomaps~\cite{hornung2013octomap}, it can be manageable.

Marginal probability computation on each cube is a direct application of Bayes
rule\footnote{$\Pr(A~|~B)=\frac{\Pr(B~|~A)\Pr(A)}{\Pr(B)}$} (a Bayes filter)
with a \textit{logit} function for better faster computaion~\cite{thrunprob}:

\begin{equation}
l_n(d) = l_{n-1}(d)+ \text{\textbf{inverse\_sensor}}(d,z_n) - l_0(d)
\end{equation}

Here $l_n(d)$ is the \textit{logit} transformation of $p_n(d)$ the nth estimate
for $p(d)$, after all previous $n$ sensor measuremnts $\mathbf{z}_n$, including
the last one $z_n$.

\begin{equation*}
l_n(d) = \log\frac{p_n(d)}{1-p_n(d)} 
\end{equation*}
\citet{Pagac1998} - Evidencial Theory


\subsubsection{Elevation Maps}
\citet{Coiras2007} - 2.5D with properties

\subsubsection{Multi Layer Surface - MLS}

\citet{Schwendner2013} - Discrete Map formulations


\subsection{Map of Features}
\citet{Ribas2006} - line extraction


\subsection{Continuous Map}
\citet{gan20093d} - Gaussian Process
\citet{ramos2016hilbert} - Hilbert Maps