\section{Inverse Sonar Model}

Methods that do not use Expectation-Maximization (EM), as some of those
described in sections \ref{sss:3dgrid}, usually require some form of inversion
of the sensor model. This is a way to characterize the environemnt from a sensor
response.

In the context of grid maps, it appear as a conditional probability on the
measurement, i.e.
$\text{\textbf{inverse\_sensor}}(d,z_n)=\Pr(\coord{m}(d)=1~|~z_n)$ in equation
\ref{eq:gridoccupancy}. For sonars, it is generally considered as a constant
value for grid elements inside an occupied bin (see figure \ref{fig:bins}) and
another for those outside\cite{thrunprob}. \citet{thrunprob} also creates an
inverse model by training a machine learning algorithm with generated responses.

The difference between an occupied bin and an empty one can be a simplistic
bin-value threshold~\cite{ribas2010underwater,Moravec1985,thrunprob} or some
more complex procedure, e.g. histograms. For Hilbert Maps, only procedures for
laser are already discribed, thus here a strategy is proposed for sonar
responses. After defining the empty and full bins, there are three main
differences between sonars and lasers: Sonars bins are volumetric; Lasers have a
definite hit point, a full bin only means that there is something somewhere
inside the bin; Sonars may have mutiple full bins, on a beam, while lasers have
only on hit.

Given the sonar/laser differences, it is suggested:
\begin{enumerate}[I]
  \item Sample the empty beam volume between the sonar and the first full bin.
  \item Sample all full bin volumes, with a possible different sample density. 
\end{enumerate}

A beam volume is a frustrum of a sphere, a region between the simple concept of
apperture angles (see section \ref{sss:bearing}) and within a range of distances
given by the bin. The sample density used was uniform on the volume, because
it is a guide for the expected echo point within the bin and there is no obvious
privileged point.

The rationale behind ignoring all empty bins after the first full bin is to
avoid considering shadow zones as empty regions, i.e. acoustic shadows of the
first echo. As empty regions means no echo anywhere inside the bin, they are
actually the most important information provived by the sonar. Each sample from
the empty beam volume is evaluated to the feature map and passed to the learning
algorithm. However, full bins indicate a echo somewhere within the bin. Thus
a whole colection of samples from a full bin are embedded to the Hilbert Space
as a distribution~\cite{song2013kernel,ramos2016hilbert}:

\begin{equation}
\label{eq:embedding}
\hat{\varphi}(\mathbb{P})=\frac{1}{n}\sum_{i=1}^n\hat{\varphi}(x_i)
\end{equation}

Where $\mathbb{P}$ is the original distribution from where $n$ samples $x_i$ are
drawn. In the learning setting, this $\hat{\varphi}(\mathbb{P})$ can be
considered more than once, if the certainty is higher, as it will be commented
in the next section.
