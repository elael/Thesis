\section{Inverse Sonar Model}

Methods that do not use Expectation-Maximization (EM), as some of those
described in sections \ref{sss:3dgrid}, usually require some form of inversion
of the sensor model. This is a way to characterize the environemnt from a sensor
response.

In the context of grid maps, it appear as a conditional probability on the
measurement, i.e.
$\text{\textbf{inverse\_sensor}}(d,z_n)=\Pr(\coord{m}(d)=1~|~z_n)$ in equation
\ref{eq:gridoccupancy}. For sonars, it is generally considered as a constant
value for grid elements inside an occupied bin (vide figure \ref{fig:bins}) and
another for those outside\cite{thrunprob}. \citet{thrunprob} also creates an
inverse model by training a machine learning algorithm with generated responses.

The difference between an occupied bin and an empty one can be a simplistic
threshold~\cite{ribas2010underwater,Moravec1985,thrunprob} or some more complex
procedure. For Hilbert Maps, after defining the empty and full bins, 