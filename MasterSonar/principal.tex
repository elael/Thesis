\documentclass[msc,numbers]{formating/coppe}
\usepackage{amsmath,amssymb}    % simbolos macos providos pela AMS
%\usepackage{hyperref}
%\usepackage[breaklinks]{hyperref}
\usepackage[linktocpage=true,a4paper]{hyperref}
\usepackage{float}
\usepackage{subfig}

\usepackage[utf8]{inputenc}

%\usepackage[brazil]{babel}
\usepackage{ae}
%\usepackage[latin1]{inputenc}
%%% CBA
\usepackage{graphicx}
%\usepackage{graphicx, subfigure}
\usepackage{psfrag}
\usepackage{epsfig}
%
\usepackage{latexsym}
\usepackage{fancybox,fancyhdr}
\usepackage{xcolor}
\usepackage{pstricks}
%%%
%\usepackage[hypertex]{hyperref} %Make sure it comes last of your loaded packages
\hypersetup{
  verbose,
  %dvips={true},
  plainpages=false,
  bookmarks=true,
  colorlinks=true,
  linkcolor=blue,
  anchorcolor=red,
  citecolor=green,
  filecolor=magenta,
  menucolor=red,
  pagecolor=red,
  urlcolor=blue
}
%%%
%\usepackage{indentfirst}        % indenta os primeiros par�grafos
%\usepackage{color}              % para letras e caixas coloridas
%\usepackage{makeidx}            % �ndice remissivo
%\usepackage{setspace}           % para a dist�ncia entre linhas
%\usepackage{fancyhdr,lastpage}
%\usepackage{mathrsfs}
%\usepackage{marvosym}
%\usepackage{array}
%\usepackage{nicefrac}
%\usepackage{xspace}
%\usepackage{nomencl}
%\usepackage{lscape}
%\usepackage{longtable}
%\usepackage{rotating}
%\usepackage{multirow}
%\usepackage{graphics}

% Macros
\newtheorem{teorema}{Theorem}
\newtheorem{simulation}[teorema]{Simula��o}%\newcommand{\tr}{\mbox{tr}}
\newtheorem{traj}[teorema]{Trajet�ria}%\newcommand{\tr}{\mbox{tr}}
%\newcommand{\ve}{\mbox{vec}}
%\newcommand{\h}{\mbox{herm}}
%\newcommand{\posto}{\mbox{posto}}
%\newcommand{\diag}{\mbox{diag}}
%\newcommand{\sign}{\mbox{sign}}
%\newcommand{\R}{\mathbb{R}}

\newcommand{\Appendix}{\par
  \setcounter{chapter}{0}
  \setcounter{section}{0}
  \setcounter{subsection}{0}
  \renewcommand{\chaptername}{\appendixname}
  \renewcommand{\thechapter}{\Alph{chapter}}
  \renewcommand{\thesection}{\Alph{chapter}.\arabic{section}}
  \renewcommand{\thesubsection}{\Alph{section}.\arabic{section}.\arabic{subsection}}
  \renewcommand{\theequation}{\Alph{chapter}{\arabic{equation}}}
}

\makelosymbols
\makeloabbreviations
\makeindex

\begin{document}

\title{Inversa Filtrada: uma Solo Alternativa para a Cinemtica Inversa de Manipuladores Robicos}
\foreigntitle{Filtered Inverse: an Alternative Solution to Inverse Kinematics of Robotic Manipulators}
\author{Lucas Vares}{Vargas}
\advisor{Prof.}{Ramon Romankevicius}{Costa}{D.Sc.}
\advisor{Prof.}{Antonio Candea}{Leite}{D.Sc.}
%\examiner{Prof.}{Ramon Romankevicius}{Costa}{D.Sc.}
%\examiner{Prof.}{Ant�nio Candea}{Leite}{D.Sc.}
%\examiner{Prof.}{Amit Bhaya}{Ph.D.}

\examiner{Prof.}{Ramon Romankevicius Costa}{D.Sc.}
\examiner{Prof.}{Antonio Candea Leite}{D.Sc.}
\examiner{Prof.}{Amit Bhaya}{Ph.D.}
\examiner{Prof.}{Marco Antonio Meggiolaro}{Ph.D.}
\department{PEE} \date{03}{2013}
\keyword{Cinematica Inversa} \keyword{Singularidades}
\keyword{Manipuladores Roticos} \keyword{Inversa Filtrada}
\maketitle
\frontmatter

\dedication{Mammy.}

\chapter*{Agradecimentos}
%%%
A Deus,
%%% ORIENTADORES

%%% PROFESSORES

%%% LABCON LEAD


%%% AMIGOS QUE AJUDARAM

%%% AMIGOS

%%% Ze LUIS
%Ao professor Jos� Lu�s da Silveira,
%

%%%

%%%
% COPPE
%%%

\begin{abstract}
%%%%%%%%%%%%%%%%%%%%%%%%%%%%%%%%%% ABSTRACT
\end{abstract}

\begin{foreignabstract}
%%%%%%%%%%%%%%%%%%%%%%%%%%%%%%%%%% ABSTRACT
\end{foreignabstract}

\tableofcontents
\listoffigures
\printloabbreviations
\printlosymbols

%\pagestyle{plain}
\chapter*{Abreviaturas}
\noindent
{\small
\begin{tabular}{lp{12cm}}
{\bf{IF:}} & \raggedright {Inversa Filtrada} \\ {\emph{(Filtered Inverse)}}\tabularnewline
{\bf{DoF:}} & \raggedright {Graus de liberdade} \\ {\emph{(Degrees of Freedom)}}\tabularnewline
{\bf{DH:}} & \raggedright {Denavit-Hartenberg} \\ \tabularnewline
%{\bf{SPR}} & \raggedright {Estritamente Real Positivo} \\ {\emph{(Strictly Positive Real)}} \tabularnewline
%{\bf{WSPR}} & \raggedright {W-Estritamente Real Positivo} \\ {\emph{(W-Strictly Positive Real)}} \tabularnewline
%{\bf{MRC}} & \raggedright {Controle por Modelo de Refer�ncia} \\ {\emph{(Model Reference Control)}} \tabularnewline
%{\bf{MRAC}} & \raggedright {Controle Adaptativo por Modelo de Refer�ncia} \\ {\emph{(Model Reference Adaptive Control)}} \tabularnewline
%{\bf{SMC}} & \raggedright {Controle por Modos Deslizantes} \\ {\emph{(Sliding Mode Control)}} \tabularnewline
%{\bf{VSC}} & \raggedright {Controle a Estrutura Vari�vel} \\ {\emph{(Variable Structure Control)}} \tabularnewline
%{\bf{VS-MRAC}} & \raggedright {Controle Adaptativo por Modelo de Refer�ncia a Estrutura Vari�vel} \\ {\emph{(Variable Structure MRAC)}} \tabularnewline
%{\bf{UVC}} & \raggedright {Controle Vetorial Unit�rio} \\ {\emph{(Unit Vector Control)}} \tabularnewline
%{\bf{LTI}} & \raggedright {Linear Invariante no Tempo} \\ {\emph{(Linear Time Invariant)}} \tabularnewline
%{\bf{LMI}} & \raggedright {Desigualdade Matricial Linear} \\ {\emph{(Linear Matrix Inequality)}} \tabularnewline
%{\bf{HFG}} & \raggedright {Ganho de Alta Frequ�ncia} \\ {\emph{(High Frequency Gain)}} \tabularnewline
\end{tabular}
}
\mainmatter
%%%%%%%%%%%%%%%%%%%%%%%%%%%%%%%%%%%%%%%%%%
%%%%%%%%%%%%%%%%%%%%%%%%%%%%%%%%%%%%%%%%%%
\chapter{Introduction}

%% SrJ: add explanations about profiling vs. imaging: size of beam and
%% SrJ: implications (ambiguities)

Underwater mapping and simulation are dual processes, while the latter produce
sonar responses for a given environment, the former use these responses to infer
the surroundings. As such simulation is a flexible way of generating data with a
known ground truth to test a mapping algorithm. However, to archive a correct
underwater simulation algorithm, simplifying assumptions on sound physics
and environmental characteristics are necessary, as well as a definition of the
sonar type being modeled.
%SLAM 

Profilings and imaging sonars are two classes of sonars whose differences lie
in the aperture of their sound beams. Profiling sonars have a narrow sound beam
and they are considered the laser scanner analog for underwater mapping,
even though profilings still have much wider beam than lasers. The simplest
approaches to underwater 3D mapping focus on applying laser scanner techniques
to profiling sonars, e.g. point cloud reconstruction.
% % SrJ: they don't really "give more information". They cover more space per %
% SrJ: beam, but the information is more ambiguous
 On the other hand, imaging sonars are usually
cheaper and have a wider sound beam, covering more space at the expense of
having a more ambiguous response. Thus, the choice of using imaging
sonars for mapping comes with the challenge of overcoming its measurement's
uncertainty.
%and benefiting from its extra information is a win-win strategy, the .
% % SrJ: "a win-win strategy" means that there are two parties ... I only see
% one

%Mapping is subject to different interpretations,


% % SrJ: apart from poor wording, the following is off the point. What you
% really
% % SrJ: want to stress is that there are various map representations. There are
% % SrJ: SLAMs on dense 3D maps / 2.5D ! Not every SLAM is feature-based, and a
% % SrJ: lot of feature-based SLAMs can generate dense maps as a byproduct
Besides stipulating a sonar type, the meaning of mapping ought to be narrowed
down. It is possible to generically define mapping as the process of gathering
multiple sensor data to characterize the surroundings. However, how this
characterization might be represented is dependent on the application.

A SLAM (\textbf{S}imultaneous \textbf{L}ocalization \textbf{a}nd
\textbf{M}apping) system has no intrinsic need for a human readable map.
In such a system, it could be interesting to store the map information only
through its most representative features, but even for SLAM that is not always
the case. It is often implemented as a grid with empty/full cells or even as a
continuous map.

The mapping of underwater environments is not just a part of a SLAM system. It
has importance on its own, it can be used for humans to visualize things that
could not be seeing otherwise. If the map is to be seen by a human it should
store and merge information about the environment, so that it can be displayed
as a usual map, 3D or 2D depending on the case. This representation also guides
how the data could be stored, e.g. if it wants to show a surface, it can be
stored as an elevation map, or if one wants to see a 3D object it can be stored
as a point cloud, a 3D grid, a continuous map, etc.

\section{Purpose and Significance}
%\section{Motivation}

%% SrJ: OK, drop the whole "SLAM" thing. Really, just mention it briefly in the
%% SrJ: beginning of the introduction and STOP
%%
%% SrJ: you're not doing SLAM, and mapping is important. Period.

%% SrJ: AFAIK, you won't run your algorithm on any ESBR data .... I wouldn't
% % SrJ: mention it here
% The mapping of underwater environments is not just a part of a SLAM system. It
% might have importance on its own, it can be used for humans to visualize things
% that could not be seeing otherwise. 

In the ROSA (\textit{Robô para Operações de Stoplogs Alagados}) project,
developed by LEAD/COPPETEC for ESBR (Engenharia Sustentável do Brasil), one of
the goals is to make a reconstruction of the hydroelectric power plant turbine
entrance. It should spot any underwater debris that could block the lowering of
stoplogs\footnote{long rectangular timber beams stacked to block water
flow.} and cause delays or even accidents.
Interestingly, the stoplog setup has characteristics that make it appropriate
for sonar mapping. It has a lifting beam for inserting stoplogs into
water that can act as stable fixation point for any sonar structure and
provide a good means of localization. Well placed high-end sonars could probably
scan such a environment, but they are expensive.

Mechanical imaging sonar is a more affordable type of sonar, however it suffers
from imprecise measurements caused by its wide sound beam. This works aims to
provide a method to map an environment using imaging sonars. It extends a recent
developed continuous map technique (Hilbert maps~\cite{ramos2016hilbert}) by
applying it to sonar responses. It also implement a simulator with a trade-off
between having simplifying assumptions and being as complete as possible for
imaging sonars, justifying the choices based on first physical principles and
other advanced simulation techniques.


% 
% 
% %% SrJ: this is really ... out of context
% It is also a integral part ROV's, where it gives feedback to the operator for
% him to know where it is or/and what he is doing, especially because cameras do
% %% SrJ: I don't get that last sentence.
% not have a very useful range. And glaringly its automated counterpart (AUV's) as
% a requisite for SLAM.

\section{Review of the Literature}

********

1)Comment other simulators and what they miss (multipath or forwardlooking,
generally).
\citet{Coiras2009} - Simulation SAS
\citet{coiras2009gpu} - GPU based SAS

2)Cite first principles confusion on references (Intensity/RMS Acoustic
Pressure), Lambertian reflection, etc\ldots

3)Comment other mappings and what they miss (3D sonar or computability usually).
\\\\
******** 



\section{Objectives}

%% SrJ: try to be more bullet-pointy
%% SrJ: you already made your point about "that can be done in many ways" focus
%% SrJ: on what this thesis does.

% %, which technical details are based on Tritech's Micron sonar

The objectives of this work are the development of a simulator and a mapping
system for mechanical imaging sonars, thus being able to validate the hypothesis
that these sonars can be used for mapping.
The simulator will be able to receive a description of a general environment and calculate the expected response of a high-frequency imaging sonar. Its output
shall exhibit common sonar features as multipath effects, noise and beamwidth
uncertainty. The response will be presented as polar plot similar to those used
by real sonars.

The mapping system will receive sonar measurements, typically the simulator's
output, and generate a Hilbert map representation of the continuous occupancy
map, by apply a simple embedding of sonar response method envisioned by the
author.
The map has to match to the simulator ground truth avoiding inconsistencies
caused by the sonar's wide beam. The Hilbert map representation is, technically,
just a vector that encodes 3D occupancy maps, however the information of this 3D
map will be displayed as 2D cross-sections for better readability.


% The general objective is to map an underwater environment.
% But that can be done in many ways, for this thesis the main goal is to implement
% optimization of the visible surface based on the expected response of a
% sonar beam in a given direction and fuse multiple views using binary bayesian
% filters. 
% 
% The output of the whole process will be a 3D human-redable map of the scanned
% environemnt. The optimization step might, also, supply information about the
% surface material, specifically its reflectance. A technique based on previously
% articles.
% 
%  Fuse all the data on a octree structure using a robotic framework,
% named ROCK, and implement a visualization for the reconstruct underwater map.
% The data source shall be a imaging sonar mounted on a pan-tilt unit, so to
% provide the sonar extra degrees of freedom.

%  But to get there it is important to describe some facts first.
%  
% As stated before, profiling sonar try to mimic laser scanner, but using sound.
% So most of the approaches on 3D underwater SLAM focus on applying laser scanner
% techniques to profiling sonars, e.g. point cloud reconstruction. On the other
% hand, imaging sonars are usually cheaper and gives more information per sound
% beam. So having the possibility of using imaging sonar and benefiting from its
% extra information is a win-win strategy.
% 
% Besides the definition of the sonar, one should carefully look into the meaning
% of mapping, because, it can be interpreted in different ways, depending on the
% context.
% Taking what they all have in common, it is possible to generically define
% mapping as being the process of received environmental data to characterize
% the surroundings. In the understanding of characterization is where the
% difference lays, how the environs are going to be represented is dependent on
% the application.
% 
% On a SLAM system there is no intrinsic need for a human readable map. In such a
% system, it may be more interesting to store the map information, the
% \"characterization\", only through its most representative features, if that is
% what matters for the localization procedure.
% 
% Still, if the map is to be seen by a human it should store more information
% about the environment, so that it can be displayed as a usual map, 3D or 2D
% depending on the case. This representation also guides how the data could be
% stored, e.g. if it wants to show a surface, it can be stored as an elevation map,
% or if one wants to see a 3D object it can be stored as a point cloud.
% 
% Using de aforementioned ideas, the main goal of the thesis reported here is to
% implement state-of-the-art techniques for mapping, using binary bayesian
% filters. Fuse all the data on a octree structure using a robotic framework,
% named ROCK, and implement a visualization for the reconstruct underwater map.
% The data source shall be a imaging sonar mounted on a pan-tilt unit, so to
% provide the sonar extra degrees of freedom.
% 
% Optimization of the visible surface for based on the expected response of a
% sonar beam in a given direction will also be attempted. It might supply
% information about the surface material, specifically its reflectance. A
% technique based on previously articles.



\section{Methodology}
%\section{Metodology and Expected Results}
%But to get there it is important to describe some facts first.
 


% Using de aforementioned ideas, the main goal of the thesis reported here is to
% implement state-of-the-art techniques for mapping, using binary bayesian
% filters. Fuse all the data on a octree structure using a robotic framework,
% named ROCK, and implement a visualization for the reconstruct underwater map.
% The data source shall be a imaging sonar mounted on a pan-tilt unit, so to
% provide the sonar extra degrees of freedom.
% 
% Optimization of the visible surface for based on the expected response of a
% sonar beam in a given direction will also be attempted. It might supply
% information about the surface material, specifically its reflectance. A
% technique based on previously articles.


This work is divided into two parts.

The sonar model definition starts with a compilation on the description of
sonar, physical properties of sound waves in water, reflection, sonar
directional gain and sources of noise. Those are used to select a simulation
technique and model two different environment, a simple and a complex
structure, one to feed to the mapping algorithm and another to explore more
advanced acoustic features, e.g. multipath, directional gain. The simulation
results for both environments are then analyzed for those typical sonar
features.

The second part is related to mapping. An introductory chapter presents
the mathematical concepts used, followed by another with discussions on the
difficulties of 3D reconstruction and its methods. The latter includes description
of the most common and standard state-of-the-art techniques, with comments on
some alternative works, and deeper details of Hilbert maps. Hilbert maps
implementation and results, for one of the simulated environments, are
displayed at the end of the chapter.

% The second branch deal with the map filling over a discretized space, based on
% a Binary Bayesian Filter implementation. That is the standard state-of-the-art
% technology for mapping \cite{thrunprob}.


% The second branch deal with the map filling, basically the Binary Bayesian
% Filter implementation. A review on Bayesian Filtering is scheduled before the
% coding writing to be done on the robotics framework, ROCK. The implementation
% will make use of the Octotree data structure, through the Octomap library, to
% store the map.

% The integration of the sonar model with the Bayesian Filter give the means to
% process sonar data. So data acquired on the LNDC/UFRJ tank (Laboratório de
% Ensaios Não Destrutivos, Corrosão e Soldagem - which loosely translated means
% Non-Destructive Testing, Corrosion and Welding Laboratory) and on the Jirau
% Power Plant, by means of the ROSA/COPPETEC project, will be processed and
% compared to the tank and power plant entry layout, respectively.
%  
% In a more complex endeavor, which is the third branch, a theoretical derivation
% for the optimization of 2D surfaces embed on 3D environments will take place.
% It uses the sonar response as expectation and, instead of having a fixed
% relation between the measured environment and the sonar response, aim to better
% infer the underlying geometry of the surrounds and also retrieve some
% information about the material's reflectivity properties.
% This optimization algorithm will then be integrated into the Bayesian Filter. The
% same data processed by the combination ``Filter + Sonar Model'' shall now be
% processed by ``Filter + Optimization''.
%  
% Based on the literature, even with not much similar works, the reconstruction
% with \textit{a priori} sonar model will probably experience problems when
% reconstructing corners, shallow angle surfaces, very complex (intense multipath)
% or on highly noise environments. For the optimization, it shall encounter
% similar issues, but a less accentuated shallow angle quality degradation and an
% overall less blurry reconstruction.

\section{Work Structure}

\textbf{Chapter 1}\quad Motivation and general description of the thesis.
\\
\textbf{Chapter 2}\quad Description of sonar models and their working
principle. Development of simulation logic from acoustics. Review of
simulations techniques on the literature. Implementation and results of a
simulator based on a ray theory algorithm.
\\
\textbf{Chapter 3}\quad Presentation of the mathematical structure necessary for
understanding Hilbert Maps.
\\
\textbf{Chapter 4}\quad Review of mapping techniques on the literature.
Detailed description of Hilbert Maps and a proposal of sonar response embbeding.
Implementation and results for a box-like environment, from a simulation of
chapter 2.
\\
\textbf{Chapter 5}\quad Comparison between simulation ground truth and
reconstructed environment. Suggestion of next steps to improve both simulation
and mapping.
% 
% ********
% 
% 1) First principles -> Ray Theory (other techniques along the way)
% 
% 2) Enronments and Materials
% 
% 3) Simulation and Results
% 
% 4) Math to Mapping
% 
% 5) Mapping concept
% 
% 6) Mapping implementation and results
% 
% 7) Conclusion / Comparison
% \\\\
% ********

\chapter{\label{chap::cinematica}Cinem�tica}

..


\section{Posi��o e orienta��o de um corpo r�gido}

..

\section{\label{sec::cd}Cinem�tica direta}

..

\section{\label{sec::dif}Cinem�tica diferencial}

..

\section{Cinem�tica inversa}

..
\chapter{\label{chap::sing}Singularidades e mtodos alternativos}

..

\section{SVD e manipulabilidade}

..

\section{Singularidades em rcos}

..

\section{Algoritmo DLS}

..

\section{Algoritmo FIK}

..



	
\include{chap4_m}
\chapter{Conclus�es e Trabalhos Futuros}

\section{Conclus�es}

\begin{itemize}

	\item Este trabalho apresenta um algoritmo alternativo para o c�lculo da cinem�tica inversa de rob�s manipuladores, que pode ser interpretado como uma inversa filtrada. A ideia b�sica do algoritmo proposto consiste em calcular a inversa dinamicamente, com rapidez determinada pelo condicionamento da matriz. Foram apresentados diversos resultados de simula��o que ilustram o bom desempenho do algoritmo proposto. %Resultados de simula��o para dois algoritmos consolidados na literatura (DLS e FIK) tamb�m s�o apresentados para an�lise e compara��o.
	\item Neste trabalho, o algoritmo da inversa filtrada � generalizado para matrizes retangulares com a utiliza��o de uma lei de atualiza��o composta, garantindo ao m�todo uma forma geral independente das dimens�es da matriz considerada.
	\item A utiliza��o do algoritmo proposto e o ajuste de ganhos permite a prioriza��o de um dos objetivos de controle quando estes s�o inalcan��veis (e.g., posi��o e orienta��o). Um problema de dimens�o ampliada tamb�m pode ser definido, considerando uma restri��o adicional cuja prioridade est� da mesma forma relacionada ao ajuste de ganhos. S�o propostas duas fun��es objetivo, associadas a limite de juntas e desvio de obst�culos.
	\item Comparado com outros algoritmos descritos na literatura, uma vantagem do algoritmo proposto � sua facilidade de sintonia pois possui apenas um par�metro de ajuste: um ganho de adapta��o. Este ganho est� diretamente relacionado ao desempenho do m�todo. Do ponto de vista computacional, o m�todo proposto se mostra eficiente por n�o requerer a invers�o da matriz Jacobiana, assim como o c�lculo de valores singulares.
	\item O desenvolvimento de um ambiente de realidade virtual para \emph{Simulink} tamb�m constitui uma contribui��o deste trabalho, permitindo a visualiza��o em tempo real do movimento da estrutura mec�nica.
\end{itemize}

%\begin{itemize}
%	\item Neste trabalho s�o propostos dois m�todos de controle adaptativo, aplicados a manipuladores rob�ticos com cinem�tica incerta, em que os par�metros da estrutura mec�nica s�o aproximados ou at� desconhecidos.
%	\item O m�todo de identifica��o param�trica, ou m�todo de adapta��o indireta, se baseia na identifica��o dos par�metros incertos atrav�s de um estimador e a utiliza��o destes, pelo {\it Princ�pio da Equival�ncia Certa}, no c�lculo das vari�veis de controle.
%	\item O m�todo de adapta��o direta da lei de controle apresentou restri��es a sua aplica��o pois algumas condi��es devem ser satisfeitas, como a exist�ncia de uma parametriza��o linear da vari�vel de controle.
%	\item Simula��es foram realizadas para o m�todo de identifica��o param�trica, evidenciando o desempenho pouco satisfat�rio no rastreamento de trajet�rias com a utiliza��o de par�metros aproximados e a superioridade de um sistema de controle com adapta��o. Este foi capaz de estimar r�pida e corretamente os par�metros incertos do manipulador, mesmo com a presen�a de ru�dos.
%	\item O desenvolvimento de um ambiente de realidade virtual constitui outra grande contribui��o deste trabalho, podendo ser utilizado para a visualiza��o do movimento do manipulador considerado e, portanto, para o diagn�stico de qualquer poss�vel falha no desenvolvimento de algum t�pico de controle, sem maiores riscos materiais.
%\end{itemize}

\newpage
\section{Propostas de Trabalhos Futuros}

Com o intuito de aprimorar a pesquisa desenvolvida, no que se segue s�o apresentados t�picos para trabalhos futuros:
\begin{itemize}
	\item An�lise de uma proposta alternativa $\dot{\bf q} = {\boldsymbol \Theta}_p {\bf J}^T {\boldsymbol \nu}$, onde ${\boldsymbol \Theta}_P$ � a inversa filtrada de ${\bf J}^T{\bf J}$. Esta an�lise deve incluir as vantagens da utiliza��o de uma lei de atualiza��o composta, uma compara��o das matrizes ${\boldsymbol \Theta}_P$ e ${\boldsymbol \Theta}{\boldsymbol \Theta}^T$, onde ${\boldsymbol \Theta}$ � a inversa filtrada de ${\bf J}$, e o estudo do condicionamento do bom desempenho do m�todo � escolha inicial ${\boldsymbol \Theta}_P(0)$.
	\item Inclus�o de um termo \emph{feedforward} na lei de atualiza��o. O termo ${\bf F}_{\boldsymbol \Theta} = -{\boldsymbol \Theta} \dot{\bf J} {\boldsymbol \Theta}$ foi considerado durante a pesquisa, por�m n�o foi inclu�do na proposta final para a lei de atualiza��o de ${\boldsymbol \Theta}$, apenas proporcional �s matrizes de erro � esquerda e � direita. Para o caso escalar, a utiliza��o do termo correspondente $f_\theta = - \theta^2\dot{k}$ ($\dot{\theta}$ para $\theta = k^{-1}$) permite que o erro escalar $S$ permane�a nulo ap�s o transit�rio. Desta forma, a trajet�ria no plano $\theta \times k$ se desenvolve na hip�rbole onde $S = 0$, impedindo que o ganho $k$ passe por $0$ e, portanto, troque de sinal. No caso multivari�vel, a utiliza��o do termo ${\bf F}_{\boldsymbol \Theta}$ resulta no c�lculo exato da matriz inversa (ap�s o transit�rio), impedindo que a matriz Jacobiana passe por um ponto de singularidade, principal motiva��o deste trabalho. Por outro lado, sua considera��o permite um melhor desempenho para matrizes bem condicionadas e afastadas de singularidades, sugerindo o chaveamento deste termo como t�pico futuro de pesquisa.
	\item Realiza��o de ensaios experimentais. A implementa��o pr�tica do algoritmo proposto � de grande relev�ncia e deve ser considerada em trabalhos futuros. A aplica��o destes estudos de caso deve ser acompanhada de uma an�lise da aproxima��o discreta da inversa filtrada, %(\emph{backward}, \emph{forward} ou \emph{trapezoidal}),
de seu custo computacional e da influ�ncia do per�odo de amostragem no desempenho do m�todo proposto. Durante a pesquisa, resultados experimentais foram obtidos para o problema de posi��o e orienta��o de um manipulador planar 3R, por�m atrav�s da proposta escalar para $k = sin(q_2)$.
	\item Aplica��o em t�picos de controle adaptativo. Um t�pico de pesquisa a ser desenvolvido consiste na utiliza��o da abordagem proposta para solucionar o problema de cinem�tica inversa de manipuladores com incertezas em seu modelo cinem�tico (e.g., rob� manipulando um objeto de tamanho desconhecido com �ngulo desconhecido). Em geral, uma solu��o adaptativa indireta utiliza a invers�o da matriz Jacobiana estimada, cuja invertibilidade pode n�o ser garantida.
\end{itemize} 
\backmatter
\bibliographystyle{formatin/coppe-unsrt}
\bibliography{biblio}

\appendix

\chapter{\label{cbasic}Conceitos sicos}

\section{Propriedades da f}




\end{document}
