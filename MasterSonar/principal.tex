\documentclass[msc,numbers]{formating/coppe}
\usepackage{amsmath,amssymb}    % simbolos macos providos pela AMS
%\usepackage{hyperref}
%\usepackage[breaklinks]{hyperref}
\usepackage[linktocpage=true,a4paper]{hyperref}
\usepackage{float}
\usepackage{subfig}

\usepackage[utf8]{inputenc}

%\usepackage[brazil]{babel}
\usepackage{ae}
%\usepackage[latin1]{inputenc}
%%% CBA
\usepackage{graphicx}
%\usepackage{graphicx, subfigure}
\usepackage{psfrag}
\usepackage{epsfig}
%
\usepackage{latexsym}
\usepackage{fancybox,fancyhdr}
\usepackage{xcolor}
\usepackage{pstricks}
%%%
%\usepackage[hypertex]{hyperref} %Make sure it comes last of your loaded packages
\hypersetup{
  verbose,
  %dvips={true},
  plainpages=false,
  bookmarks=true,
  colorlinks=true,
  linkcolor=blue,
  anchorcolor=red,
  citecolor=green,
  filecolor=magenta,
  menucolor=red,
  pagecolor=red,
  urlcolor=blue
}
%%%
%\usepackage{indentfirst}        % indenta os primeiros par�grafos
%\usepackage{color}              % para letras e caixas coloridas
%\usepackage{makeidx}            % �ndice remissivo
%\usepackage{setspace}           % para a dist�ncia entre linhas
%\usepackage{fancyhdr,lastpage}
%\usepackage{mathrsfs}
%\usepackage{marvosym}
%\usepackage{array}
%\usepackage{nicefrac}
%\usepackage{xspace}
%\usepackage{nomencl}
%\usepackage{lscape}
%\usepackage{longtable}
%\usepackage{rotating}
%\usepackage{multirow}
%\usepackage{graphics}

% Macros
\newtheorem{teorema}{Theorem}
\newtheorem{simulation}[teorema]{Simula��o}%\newcommand{\tr}{\mbox{tr}}
\newtheorem{traj}[teorema]{Trajet�ria}%\newcommand{\tr}{\mbox{tr}}
%\newcommand{\ve}{\mbox{vec}}
%\newcommand{\h}{\mbox{herm}}
%\newcommand{\posto}{\mbox{posto}}
%\newcommand{\diag}{\mbox{diag}}
%\newcommand{\sign}{\mbox{sign}}
%\newcommand{\R}{\mathbb{R}}

\newcommand{\Appendix}{\par
  \setcounter{chapter}{0}
  \setcounter{section}{0}
  \setcounter{subsection}{0}
  \renewcommand{\chaptername}{\appendixname}
  \renewcommand{\thechapter}{\Alph{chapter}}
  \renewcommand{\thesection}{\Alph{chapter}.\arabic{section}}
  \renewcommand{\thesubsection}{\Alph{section}.\arabic{section}.\arabic{subsection}}
  \renewcommand{\theequation}{\Alph{chapter}{\arabic{equation}}}
}

\makelosymbols
\makeloabbreviations
\makeindex

\begin{document}

\title{Inversa Filtrada: uma Solo Alternativa para a Cinemtica Inversa de Manipuladores Robicos}
\foreigntitle{Filtered Inverse: an Alternative Solution to Inverse Kinematics of Robotic Manipulators}
\author{Lucas Vares}{Vargas}
\advisor{Prof.}{Ramon Romankevicius}{Costa}{D.Sc.}
\advisor{Prof.}{Antonio Candea}{Leite}{D.Sc.}
%\examiner{Prof.}{Ramon Romankevicius}{Costa}{D.Sc.}
%\examiner{Prof.}{Ant�nio Candea}{Leite}{D.Sc.}
%\examiner{Prof.}{Amit Bhaya}{Ph.D.}

\examiner{Prof.}{Ramon Romankevicius Costa}{D.Sc.}
\examiner{Prof.}{Antonio Candea Leite}{D.Sc.}
\examiner{Prof.}{Amit Bhaya}{Ph.D.}
\examiner{Prof.}{Marco Antonio Meggiolaro}{Ph.D.}
\department{PEE} \date{03}{2013}
\keyword{Cinematica Inversa} \keyword{Singularidades}
\keyword{Manipuladores Roticos} \keyword{Inversa Filtrada}
\maketitle
\frontmatter

\dedication{Mammy.}

\chapter*{Agradecimentos}
%%%
A Deus,
%%% ORIENTADORES

%%% PROFESSORES

%%% LABCON LEAD


%%% AMIGOS QUE AJUDARAM

%%% AMIGOS

%%% Ze LUIS
%Ao professor Jos� Lu�s da Silveira,
%

%%%

%%%
% COPPE
%%%

\begin{abstract}
%%%%%%%%%%%%%%%%%%%%%%%%%%%%%%%%%% ABSTRACT
\end{abstract}

\begin{foreignabstract}
%%%%%%%%%%%%%%%%%%%%%%%%%%%%%%%%%% ABSTRACT
\end{foreignabstract}

\tableofcontents
\listoffigures
\printloabbreviations
\printlosymbols

%\pagestyle{plain}
\chapter*{Abreviaturas}
\noindent
{\small
\begin{tabular}{lp{12cm}}
{\bf{IF:}} & \raggedright {Inversa Filtrada} \\ {\emph{(Filtered Inverse)}}\tabularnewline
{\bf{DoF:}} & \raggedright {Graus de liberdade} \\ {\emph{(Degrees of Freedom)}}\tabularnewline
{\bf{DH:}} & \raggedright {Denavit-Hartenberg} \\ \tabularnewline
%{\bf{SPR}} & \raggedright {Estritamente Real Positivo} \\ {\emph{(Strictly Positive Real)}} \tabularnewline
%{\bf{WSPR}} & \raggedright {W-Estritamente Real Positivo} \\ {\emph{(W-Strictly Positive Real)}} \tabularnewline
%{\bf{MRC}} & \raggedright {Controle por Modelo de Refer�ncia} \\ {\emph{(Model Reference Control)}} \tabularnewline
%{\bf{MRAC}} & \raggedright {Controle Adaptativo por Modelo de Refer�ncia} \\ {\emph{(Model Reference Adaptive Control)}} \tabularnewline
%{\bf{SMC}} & \raggedright {Controle por Modos Deslizantes} \\ {\emph{(Sliding Mode Control)}} \tabularnewline
%{\bf{VSC}} & \raggedright {Controle a Estrutura Vari�vel} \\ {\emph{(Variable Structure Control)}} \tabularnewline
%{\bf{VS-MRAC}} & \raggedright {Controle Adaptativo por Modelo de Refer�ncia a Estrutura Vari�vel} \\ {\emph{(Variable Structure MRAC)}} \tabularnewline
%{\bf{UVC}} & \raggedright {Controle Vetorial Unit�rio} \\ {\emph{(Unit Vector Control)}} \tabularnewline
%{\bf{LTI}} & \raggedright {Linear Invariante no Tempo} \\ {\emph{(Linear Time Invariant)}} \tabularnewline
%{\bf{LMI}} & \raggedright {Desigualdade Matricial Linear} \\ {\emph{(Linear Matrix Inequality)}} \tabularnewline
%{\bf{HFG}} & \raggedright {Ganho de Alta Frequ�ncia} \\ {\emph{(High Frequency Gain)}} \tabularnewline
\end{tabular}
}
\mainmatter
%%%%%%%%%%%%%%%%%%%%%%%%%%%%%%%%%%%%%%%%%%
%%%%%%%%%%%%%%%%%%%%%%%%%%%%%%%%%%%%%%%%%%
\chapter{Introduction}

%% SJ: add explanations about profiling vs. imaging: size of beam and
%% SJ: implications (ambiguities)

Profiling sonars is the laser scanner analog for underwater mapping,
so most of the approaches on 3D underwater SLAM (\textbf{S}imultateous
\textbf{L}ocalization \textbf{a}nd \textbf{M}apping) focus on applying laser
scanner techniques to profiling sonars, e.g.
%% SJ: they don't really "give more information". They cover more space per
%% SJ: beam, but the information is more ambiguous
point cloud reconstruction. On the other hand, imaging sonars are usually cheaper and gives more information per sound
beam. So having the possibility of using imaging sonar and benefiting from its
%% SJ: "a win-win strategy" means that there are two parties ... I only see one
extra information is a win-win strategy.

%% SJ: apart from poor wording, the following is off the point. What you really
%% SJ: want to stress is that there are various map representations. There are
%% SJ: SLAMs on dense 3D maps / 2.5D ! Not every SLAM is feature-based, and a
%% SJ: lot of feature-based SLAMs can generate dense maps as a byproduct
Besides the definition of the sonar, one should carefully look into the meaning
of mapping, because, it can be interpreted in different ways, depending on the
context.
Taking what they all have in common, it is possible to generically define
mapping as being the process of received environmental data to characterize
the surroundings. The meaning characterization is where the
difference lays, how the environs are going to be represented is dependent on
the application.

On a SLAM system there is no intrinsic need for a human readable map. In such a
system, it may be more interesting to store the map information, the
characterization, only through its most representative features, if that is
what matters for the localization procedure.

Still, if the map is to be seen by a human it should store more information
about the environment, so that it can be displayed as a usual map, 3D or 2D
depending on the case. This representation also guides how the data could be
stored, e.g. if it wants to show a surface, it can be stored as a elevation map,
or if one wants to see a 3D object it can be stored as a point cloud.

\section{Purpose and Significance}
%\section{Motivation}

%% SrJ: OK, drop the whole "SLAM" thing. Really, just mention it briefly in the
%% SrJ: beginning of the introduction and STOP
%%
%% SrJ: you're not doing SLAM, and mapping is important. Period.

%% SrJ: AFAIK, you won't run your algorithm on any ESBR data .... I wouldn't
% % SrJ: mention it here
% The mapping of underwater environments is not just a part of a SLAM system. It
% might have importance on its own, it can be used for humans to visualize things
% that could not be seeing otherwise. 

In the ROSA (\textit{Robô para Operações de Stoplogs Alagados}) project,
developed by LEAD/COPPETEC for ESBR (Engenharia Sustentável do Brasil), one of
the goals is to make a reconstruction of the hydroelectric power plant turbine
entrance. It should spot any underwater debris that could block the lowering of
stoplogs\footnote{long rectangular timber beams stacked to block water
flow.} and cause delays or even accidents.
Interestingly, the stoplog setup has characteristics that make it appropriate
for sonar mapping. It has a lifting beam for inserting stoplogs into
water that can act as stable fixation point for any sonar structure and
provide a good means of localization. Well placed high-end sonars could probably
scan such a environment, but they are expensive.

Mechanical imaging sonar is a more affordable type of sonar, however it suffers
from imprecise measurements caused by its wide sound beam. This works aims to
provide a method to map an environment using imaging sonars. It extends a recent
developed continuous map technique (Hilbert maps~\cite{ramos2016hilbert}) by
applying it to sonar responses. It also implement a simulator with a trade-off
between having simplifying assumptions and being as complete as possible for
imaging sonars, justifying the choices based on first physical principles and
other advanced simulation techniques.


% 
% 
% %% SrJ: this is really ... out of context
% It is also a integral part ROV's, where it gives feedback to the operator for
% him to know where it is or/and what he is doing, especially because cameras do
% %% SrJ: I don't get that last sentence.
% not have a very useful range. And glaringly its automated counterpart (AUV's) as
% a requisite for SLAM.

\section{Review of the Literature}

********

1)Comment other simulators and what they miss (multipath or forwardlooking,
generally).

2)Cite first principles confusion on references (Intensity/RMS Acoustic
Pressure), Lambertian reflection, etc\ldots

3)Comment other mappings and what they miss (3D sonar or computability usually).
\\\\
******** 



\section{Objetivo}

..


\section{Methodology}
%\section{Metodology and Expected Results}
%But to get there it is important to describe some facts first.
 


% Using de aforementioned ideas, the main goal of the thesis reported here is to
% implement state of the art techniques for mapping, using binary bayesian
% filters. Fuse all the data on a octree structure using a robotic framework,
% named ROCK, and implement a visualization for the reconstruct underwater map.
% The data source shall be a imaging sonar mounted on a pan-tilt unit, so to
% provide the sonar extra degrees of freedom.
% 
% Optimization of the visible surface for based on the expected response of a
% sonar beam in a given direction will also be attempted. It might supply
% information about the surface material, specifically its reflectance. A
% technique based on previously articles.


This work is divided into two parts.

The sonar model definition starts with a compilation on the description of
sonar, physical properties of sound waves in water, reflection, sonar
directional gain and sources of noise. Those are used to select a simulation
technique and model two different environment, a simple and a complex
structure. The simulation results for both environmnets are then analysed for
typical sonar features.

The second part is related to mapping. There is introductory chapter to the
mathematical concepts used, followed by another with discussions on the
difficulties of 3D reconstution and its methods. The latter includes description
most commom and standard state of the art techniques, with comments on some
alternative works, and a deeper details of Hilbert maps. Hilbert maps
implementation and results, for one of the simulated envirnoments, are displayed
at the end of the chapter.

The second branch deal with the map filling over a discretized space, based on
a Binary Bayesian Filter implementation. That is the standard state of the art
tecnology for mapping \cite{thrunprob}.


% The second branch deal with the map filling, basically the Binary Bayesian
% Filter implementation. A review on Bayesian Filtering is scheduled before the
% coding writing to be done on the robotics framework, ROCK. The implementation
% will make use of the Octotree data structure, through the Octomap library, to
% store the map.

% The integration of the sonar model with the Bayesian Filter give the means to
% process sonar data. So data acquired on the LNDC/UFRJ tank (Laboratório de
% Ensaios Não Destrutivos, Corrosão e Soldagem - which loosely translated means
% Non-Destructive Testing, Corrosion and Welding Laboratory) and on the Jirau
% Power Plant, by means of the ROSA/COPPETEC project, will be processed and
% compared to the tank and power plant entry layout, respectively.
%  
% In a more complex endeavor, which is the third branch, a theoretical derivation
% for the optimization of 2D surfaces embed on 3D environments will take place.
% It uses the sonar response as expectation and, instead of having a fixed
% relation between the measured environment and the sonar response, aim to better
% infer the underlying geometry of the surrounds and also retrieve some
% information about the material's reflectivity properties.
% This optimization algorithm will then be integrated into the Bayesian Filter. The
% same data processed by the combination ``Filter + Sonar Model'' shall now be
% processed by ``Filter + Optimization''.
%  
% Based on the literature, even with not much similar works, the reconstruction
% with \textit{a priori} sonar model will probably experience problems when
% reconstructing corners, shallow angle surfaces, very complex (intense multipath)
% or on highly noise environments. For the optimization, it shall encounter
% similar issues, but a less accentuated shallow angle quality degradation and an
% overall less blurry reconstruction.

\section{Work Structure}

********

1) First principles -> Ray Theory (other techniques along the way)

2) Enronments and Materials

3) Simulation and Results

4) Math to Mapping

5) Mapping concept

6) Mapping implementation and results

7) Conclusion / Comparison
\\\\
********

\chapter{\label{chap::cinematica}Cinemica}

..

\section{Cineca inversa}

..
\chapter{\label{chap::sing}Singularidades e mtodos alternativos}

..

\section{SVD e manipulabilidade}

..

\section{Singularidades em rcos}

..

\section{Algoritmo DLS}

..

\section{Algoritmo FIK}

..



	
\chapter{Inversa filtrada}

\ldots

\chapter{Conclues e Trabalhos Futuros}

\section{Concluses}

\section{Propostas de Trabalhos Futuros}
\backmatter
\bibliographystyle{formatin/coppe-unsrt}
\bibliography{biblio}

\appendix

\chapter{\label{cbasic}Conceitos B�sicos}

\section{Propriedades da fun��o tra�o}



\section{\label{apsec::fund}Subespa�os fundamentais}
\noindent{\bf Teorema fundamental de ortogonalidade: } O espa�o nulo $Nul({\bf A})$ e o espa�o-linha $Col({\bf A}^T)$ s�o ortogonais e subespa�os de ${\mathbb{R}^n}$. O espa�o nulo � esquerda $Nul({\bf A}^T)$ e o espa�o-coluna $Col({\bf A})$ s�o ortogonais e subespa�os de ${\mathbb{R}^m}$. %A Figura \ref{fig::subespacos}

\section{Decomposi��o de Valor Singular (SVD)}

A decomposi��o de valor singular, ou simplesmente SVD, constitui uma ferramenta extremamente poderosa para an�lise em �lgebra linear. Qualquer matriz ${\bf A}\!\in\!\mathbb{R}^{m \times n}$ pode ser fatorada em \cite{STRANG}
%
\begin{equation}
\label{eq::svd1}
{\bf A} = {\bf U}\,{\boldsymbol \Sigma}\,{\bf V}^T = \text{{\bf (ortogonal)(diagonal)(ortogonal)}}.
\end{equation}
%
Os $r$ valores singulares (denotados por $\sigma$) na diagonal de ${\boldsymbol \Sigma}\!\in\!\mathbb{R}^{m \times n}$ s�o as ra�zes quadradas de autovalores n�o nulos de ${\bf A}{\bf A}^T$ e ${\bf A}^T{\bf A}$. As colunas de ${\bf U}\!\in\!\mathbb{R}^{m \times m}$ s�o autovetores de ${\bf A}{\bf A}^T$ e as colunas de ${\bf V}\!\in\!\mathbb{R}^{n \times n}$ s�o autovetores de ${\bf A}^T{\bf A}$. As matrizes ${\bf U}$ e ${\bf V}$ fornecem bases ortonormais para todos os quatros subespa�os fundamentais apresentados em \ref{apsec::fund}:

\vskip 0.5cm
\begin{tabular}{lccl}
Primeiras & $r$ & colunas de ${\bf U}$: & {\bf espa�o-coluna} de ${\bf A}$ \\
�ltimas & $m-r$ & colunas de ${\bf U}$: & {\bf espa�o nulo � esquerda} de ${\bf A}$ \\
Primeiras & $r$ & colunas de ${\bf V}$: & {\bf espa�o-linha} de ${\bf A}$ \\
�ltimas & $n-r$ & colunas de ${\bf V}$: & {\bf espa�o-nulo} de ${\bf A}$ \\
\end{tabular}
\vskip 0.5cm

\noindent A partir de (\ref{eq::svd1}), tem-se
%
\begin{equation}
{\bf A}{\bf V} = {\bf U}\,{\boldsymbol \Sigma},
\end{equation}
%
e, portanto, quando a matriz ${\bf A}$ multiplica uma coluna ${\bf v}_j$ de ${\bf V}$, produz $\sigma_j$ vezes uma coluna de ${\bf U}$. %A raz�o $\sigma_{max}/\sigma_{min}$ � o n�mero de condicionamento de uma matriz $n$ por $n$ inversa e a disponibilidade dessa informa��o


\end{document}
