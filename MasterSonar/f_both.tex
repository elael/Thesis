\subsection{\label{sec::fjoint} Fun��o $f$ para limite das juntas}

A fun��o objetivo � definida a partir de pot�ncias pares das vari�veis de desvio das juntas normalizadas ($\Delta q_i/\zeta_i$), sendo apresentada e ilustrada abaixo. O objetivo desta fun��o � manter a vari�vel de junta $q_i$ no intervalo $[q_{id} - \zeta_i, q_{id} + \zeta_i]$.

\begin{figure}[!htp]
  \centering
  \def\JPicScale{0.26}
  \def\Lum{$\ell_1$}
  \def\Ldois{$\ell_2$}
  \def\Ltres{$\ell_3$}
  \def\Jum{Junta 1}
  \def\Jdois{Junta 2}
  \def\Jtres{Junta 3}
  \def\QD{${q}_{3d}$}
  \def\Plus{$+\zeta_3$}
  \def\Minus{$-\zeta_3$}
  \def\X{$x$}
  \def\Y{$y$}
  {\small
  \input{fig/limite_juntas.pst}
  }
  \caption{Fun��o objetivo no espa�o das juntas.}
  \label{fig::draw_joint}
\end{figure}

\begin{equation}
f(q_1,...,q_n) = \sum_{i = 1}^{n}
{\alpha_i\left(\frac{q_i - q_{id}}{\zeta_i}\right)^{2k_i}}
= \sum_{i = 1}^{n}\alpha_i \left(\frac{\Delta q_i}{\zeta_i}\right)^{2k_i}
\end{equation}
%
onde $\alpha_i \geq 0$ e $\zeta_i > 0$. Ganhos $k_i$ elevados imp�em pouca restri��o aos �ngulos das juntas, desde que permane�am dentro das faixas determinadas, e uma a��o de controle forte uma vez fora destas.

\subsection{\label{sec::fpos} Fun��o $f$ para desvio de obst�culo}

Neste caso, a fun��o objetivo � inspirada na fun��o gaussiana e definida a partir das dist�ncias dos pontos ${\bf p}_j$ ao centro ${\boldsymbol \mu}$, ponto que se deseja evitar. As curvas de n�vel s�o elips�ides, definidos pela matriz ${\bf M}_f = {\bf M}_f^T = {\bf R}_f {\bf D}_f {\bf R}_f^T > 0$. A matriz de rota��o ${\bf R}_f$ define os eixos do elips�ide e a matriz ${\bf D}_f$ apresenta elementos positivos em sua diagonal, inversamente proporcionais ao alongamento da curva de n�vel nesses eixos.
\begin{figure}[!htp]
  \centering
  \def\JPicScale{0.26}
  \def\Lum{$\ell_1$}
  \def\Ldois{$\ell_2$}
  \def\Ltres{$\ell_3$}
  \def\Jum{Junta 1}
  \def\Jdois{Junta 2}
  \def\Jtres{Junta 3}
  \def\Ct{${\boldsymbol \mu}$}
  %\def\Dx{$3\sigma_{x'}$}
  %\def\Dy{$3\sigma_{y'}$}
  \def\Pos{${\bf p} = (p_x,p_y)$}
  \def\X{$x$}
  \def\Y{$y$}
  \def\Posmi{${\boldsymbol \mu} = (\mu_x,\mu_y)$}  
  {\small
  \input{fig/obstaculo.pst}
  }
  \caption{Fun��o objetivo no espa�o operacional (posi��o).}
  \label{fig:parishilton}
\end{figure}
%
%
\begin{equation}
f({\bf p}_1,...,{\bf p}_c) = \sum_{j = 1}^{c}
{\alpha_j e^{-({\bf p}_j - {\boldsymbol \mu})^T{\bf M}_f({\bf p}_j - {\boldsymbol \mu})}}
= \sum_{j = 1}^{c}
{\alpha_j e^{-(\Delta p)'^T{\bf D}_f(\Delta p)'}},
\end{equation}
%
onde $\alpha_j \geq 0$ e $(\Delta p_j)' = {\bf R}_f^T\Delta p_j$ � a representa��o de $\Delta p_j$ no sistema de coordenadas rotacionado por ${\bf R}_f$, conforme (\ref{eq::relation}).
