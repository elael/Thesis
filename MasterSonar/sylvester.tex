\section{Equa��o de Sylvester}

Seja ${\bf A}\!\in\!\mathbb{R}^{n \times n}$, ${\bf B}\!\in\!\mathbb{R}^{m \times m}$ e ${\bf C}\!\in\!\mathbb{R}^{n \times m}$. Uma equa��o matricial linear dada por
%
\begin{equation}
\label{eq::sylvesterform}
{\bf A}{\bf X} + {\bf X}{\bf B} = {\bf C}
\end{equation}
%  
� denominada uma equa��o de Sylvester em homenagem a J. J. Sylvester, que estudou equa��es matriciais lineares descritas por
%
\begin{equation}
\sum_{i=1}^{k}{\bf A}_i{\bf X}_i{\bf B}_i = {\bf C}.
\end{equation}
%
A equa��o (\ref{eq::sylvesterform}) pode ser reescrita utilizando o operador $\text{vec}(\cdot)$ e o produto de Kronecker da seguinte forma \cite{LAUB:05}:
%
\begin{equation}
\label{eq::sylvesterap}
[({\bf I}_m \otimes {\bf A}) + ({\bf B}^T \otimes {\bf I}_n)]\text{vec}({\bf X}) = \text{vec}({\bf C}).
\end{equation}
%
Uma �nica solu��o ${\bf X}\!\in\!\mathbb{R}^{n \times m}$ existir� para (\ref{eq::sylvesterap}) se e somente se $[({\bf I}_m \otimes {\bf A}) + ({\bf B}^T \otimes {\bf I}_n)]$ for n�o-singular e, portanto, n�o apresentar autovalores nulos. Sejam $\lambda_1,...,\lambda_n$ e $\mu_1,...,\mu_m$ os autovalores de ${\bf A}$ e ${\bf B}$, respectivamente. A matriz $[({\bf I}_m \otimes {\bf A}) + ({\bf B}\, \otimes {\bf I}_n)]$, tamb�m denominada soma de Kronecker (${\bf A}\oplus{\bf B}$), apresenta $mn$ autovalores
%
\begin{equation}
\lambda_1 + \mu_1, ... , \lambda_1 + \mu_m, \lambda_2 + \mu_1, ... , \lambda_2 + \mu_m, ... , \lambda_n + \mu_1, ... , \lambda_n + \mu_m.
\end{equation}
%
Assim, uma condi��o necess�ria e suficiente para que a solu��o ${\bf X}$ seja �nica � dada por 
%
\begin{equation}
\sigma({\bf A}) \cap \sigma({\bf -B}) = {\emptyset},
\end{equation}
%
onde $\sigma({\bf A})$ e $\sigma({\bf -B})$ correspondem aos espectros de ${\bf A}$ e ${\bf -B}$, respectivamente. Isto �, existe uma �nica solu��o ${\bf X}$ para (\ref{eq::sylvesterform}) se e somente se ${\bf A}$ e ${\bf -B}$ n�o apresentarem autovalores em comum \cite{RAJ:97}. 

